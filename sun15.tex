\documentclass[twoside,11pt,nolof]{starlink}

% ? Specify used packages
% ? End of specify used packages

% -----------------------------------------------------------------------------
% ? Document identification
% Fixed part
\stardoccategory    {Starlink User Note}
\stardocinitials    {SUN}
\stardocsource      {sun\stardocnumber}


% Variable part - replace [xxx] as appropriate.
% \stardocnumber      {Draft - 15.8}
\stardocnumber      {15.8}
\stardocauthors     {D L Terrett, M J Bly}
\stardocdate        {25 February 1999}
\stardoctitle       {PGPLOT --- Graphics Subroutine Library}
\stardocversion     {v5.1.1/v5.2.0}
\stardocmanual      {User's Guide}
\stardocabstract    {PGPLOT is the \emph{de facto} standard plotting
library in Astronomy.  Starlink supports both the original `Native' version and the GKS-based `Starlink' version.  This document describes their use on
Starlink systems.}
% ? End of document identification
% -----------------------------------------------------------------------------
% ? Document specific \providecommand or \newenvironment commands.
% ? End of document specific commands
% -----------------------------------------------------------------------------
%  Title Page.
%  ===========
\begin{document}
\scfrontmatter

% ? Main text

\section{\xlabel{introduction}Introduction}
\label{introduction}

PGPLOT is a high level graphics package for plotting X \emph{versus} Y
plots, functions, histograms, bar charts, contour maps and images.
Complete diagrams can be produced with a minimal number of subroutine
calls, but control over colour, lines-style, character font,
\emph{etc},.\ is available if required.  The package was written (by Dr
T J Pearson of the Caltech astronomy department) with astronomical
applications in mind and has become a \textit{de facto} standard for
graphics in astronomy world wide.

The package exists in two versions: the original or `Native' version
which uses the low level graphics package GRPCKG which was also
written at Caltech, and a version developed by Starlink, in
collaboration with Dr Pearson, which uses GKS.  The two versions have
identical subroutine interfaces and in most cases, applications can be
moved from one version to the other simply by re-linking.

Starlink supports both the `Native' version and the `Starlink' version
which co-exist on Starlink systems.  Most packages available in the
Starlink Software Collection currently use the Starlink version but
work is currently underway to replace this with the Native version.

The current Native PGPLOT is version 5.2.0.
The current Starlink PGPLOT is based on PGPLOT Version 5.1.1.

The main source of information on using PGPLOT is the PGPLOT manual.
The manual for version 5 is still in preparation but a hypertext version
can be found at:
\begin{quote}
\url{http://astro.caltech.edu/\~tjp/pgplot/}
\end{quote}
This describes the original GRPCKG \emph{Native-PGPLOT} version but the
majority of the manual applies equally to both versions.

When using the GKS \emph{Starlink-PGPLOT} version the following
sections of the manual should be ignored and the information in this
user note used instead:

\begin{itemize}
\item Using PGPLOT (chapter 1)
\item Graphics Devices (chapter1)
\item Compiling and Running the Program (chapter 2)
\item Appendix D --- Supported Devices
\item Appendix E --- Writing a Device Handler
\item Appendix F --- Installation Instructions
\item Appendix G --- Porting PGPLOT
\end{itemize}

\section{\xlabel{native-pgplot}Native-PGPLOT}
\label{native-pgplot}

Readers interested in the Starlink-PGPLOT version should refer to
section~\latexhtml{\ref{starlink-pgplot}} %%
{\htmlref{\emph{Starlink-PGPLOT}}{starlink-pgplot}}.

\subsection{\xlabel{using_the_native_version}Using the Native version}
\label{using_the_native_version}

To link a program with the Native-PGPLOT library, use the following:

\begin{quote}
\texttt{\% f77 \emph{prog.f} -L/star/lib `pgplot\_link`}
\end{quote}

It is currently not possible to use Native PGPLOT with ADAM applications
though in future this facility will be provided.

\subsection{\xlabel{native-pgplot_device_names}Native-PGPLOT device names}
\label{native-pgplot_device_names}

Native-PGPLOT device names are described in the PGPLOT manual.  The
Native-PGPLOT device names are not the same as those used for other
Starlink graphics and it does not support the Starlink device name system.
However, substantially the same graphics device types are available
(Xwindows, Postscript).

It should be noted that the two PGPLOT systems have different Xwindows
device drivers which do not interact -- hence a program using
Starlink-PGPLOT and drawing to the `\texttt{xwindows}' device will not draw
in the same display window as a Native-PGPLOT program drawing to the
`\texttt{/XWIN}' device.

\subsection{\xlabel{native_pgplot_examples}Native-PGPLOT examples}
\label{native_pgplot_examples}

\begin{quote}
\emph{ On non-Starlink systems you may have to replace \texttt{/star} by
some other path name to locate the files referred to in this section.}
\end{quote}

Binaries of the example programs can be found in
\texttt{/star/bin/examples/pgplot}.
They can be run with a command such as:

\begin{quote}
\texttt{/star/bin/examples/pgplot/pgdemo{\emph{n}}}
\end{quote}

where \texttt{\emph{n}} is between 1 and 17, provided that they have
been installed.  (They may not have been in order to save disk space).

\section{\xlabel{starlink-pgplot}Starlink-PGPLOT}
\label{starlink-pgplot}

Readers interested in the Native-PGPLOT version should refer to
section~\latexhtml{\ref{native-pgplot}} %%
{\htmlref{\emph{Native-PGPLOT}}{native-pgplot}}.

\subsection{\xlabel{using_the_gks_version}Using the GKS version}
\label{using_the_gks_version}

There are two ways in which the GKS version of PGPLOT can be used:

\begin{enumerate}

\item As a self contained graphics package where \emph{all} graphics, including
opening and closing the workstation, is done with PGPLOT. Programs written in
this way can be run with other implementations of PGPLOT. Such programs
are linked with the command:

\begin{quote}
\texttt{\% f77 \emph{prog.f} -L/star/lib `pgp\_link`}
\end{quote}

\item To plot a picture in the current viewport of an already open GKS
workstation (see \latexhtml{section~\ref{plotting_in_the_current_viewport}} %%
{\htmlref{Plotting in the current viewport}{plotting_in_the_current_viewport}}).
\end{enumerate}

To use PGPLOT in an ADAM application, consult the
\xref{\emph{ADAM Graphics Programmer's Guide}}{sun113}{} (SUN/113).

\subsection{\xlabel{starlink-pgplot_device_names}Starlink-PGPLOT device names}
\label{starlink-pgplot_device_names}

Any graphics device supported by GKS can be used with Starlink PGPLOT
and device names are translated using the graphics name service
described in \xref{\emph{GNS -- Graphics Name
Service}}{sun57}{GKSWorkstationNames} (SUN/57).  Device names
containing a `\texttt{/}' character (such as UNIX file names) must be
surrounded by quote (`\texttt{"}') characters.

If a question mark is typed in response to the prompt from
\xref{\texttt{PGBEG}}{sun15}{PGBEG}, a list of those workstation names
defined on your system will be listed on the terminal.

On some hard copy devices the output from a PGPLOT program is a file
and some further action (such as printing the file) is required to
produce a plot.  If you are unfamiliar with a particular device,
consult \xref{SUN/83}{sun83}{}.

If one of the metafile workstations is selected the metafile can be
tailored for a particular real workstation type by appending
\texttt{/TARGET=\emph{workstation}} to the device specification.  The
default target is the monochrome A4 Postscript workstation. The
resulting metafile can be played back on any workstation but will be
tailored with respect to such things as resolution, number of colour
indices, etc. for the selected target.

The device name syntax described in the PGPLOT manual is also
supported; when using this form of device name, the device type is
specified using a GNS workstation name.

\subsection{\xlabel{starlink-pgplot_examples}Starlink-PGPLOT examples}
\label{starlink-pgplot_examples}

\begin{quote}
\emph{ On non-Starlink systems you may have to replace \texttt{/star} by
some other path name to locate the files referred to in this section.}
\end{quote}

The directory \texttt{/star/share/pgp} contains the source of a
number of example programs which demonstrate most of the features of
PGPLOT.  Binaries of the example programs can be found in
\texttt{/star/bin/examples/pgp}.

They can be run with a command such as:

\begin{quote}
\texttt{/star/bin/examples/pgp/pgdemo{\emph{n}}}
\end{quote}

where \texttt{\emph{n}} is between 1 and 14, provided that they have
been installed.  (They may not have been in order to save disk space).

\subsection{\xlabel{plotting_in_the_current_viewport} %%
Plotting in the current viewport}
\label{plotting_in_the_current_viewport}
\label{viewport}

PGPLOT can be used to plot a picture in the current viewport on an
already open GKS workstation. When used in this way, the second
argument to \xref{\texttt{PGBEG}}{sun15}{PGBEG} (normally the workstation
name) is a GKS workstation identifier (encoded as a character string).
PGPLOT then behaves as if the region of the display surface defined by
the current viewport is a complete workstation.  When
\xref{\texttt{PGEND}}{sun15}{PGEND} is called the workstation is not closed
but the state of GKS is restored to what it was at the time that
\texttt{PGBEG} was called.

PGPLOT assumes that it has exclusive control over the GKS and so the only
graphics calls allowed between \texttt{PGBEG} and \texttt{PGEND} are PGPLOT
routines and GKS inquiry routines.

The following simple example is a subroutine that uses PGPLOT to draw an X, Y
plot in an SGS zone.
\begin{small}
\begin{terminalv}
     SUBROUTINE XYPLOT (IZONE, X, Y, N, XLO, XHI, YLO, YHI, ISTAT)
*++
*
*   XYPLOT   Draw X,Y plot in an SGS zone.
*
*   Description:
*
*      Uses PGPLOT to draw an X,Y plot of the real arrays X & Y in the
*      region of the display surface defined be the specified SGS zone.
*
*   Input arguments:
*
*      IZONE    INTEGER         SGS zone identifier
*      X        REAL(N)         X values of data points
*      Y        REAL(N)         Y   "    "    "    "
*      N        INTEGER         Number of data points
*      XLO      REAL            Lower X axis limit
*      XHI      REAL            Higher X  "    "
*      YLO      REAL            Lower Y  "     "
*      YHI      REAL            Higher Y  "    "
*
*   Output arguments:
*
*      ISTAT    INTEGER         SGS status
*
*   Side effects:
*
*      The specified SGS zone is selected.
*++
     IMPLICIT NONE
     INTEGER  IZONE, N, ISTAT, PGSTAT
     REAL     X(N), Y(N), XLO, XHI, YLO, YHI

     CHARACTER*10 WKID
     INTEGER IWKID

     INTEGER PGBEG


*  Select the specified SGS zone.
     CALL SGS_SELZ(IZONE, ISTAT)
     IF (ISTAT.EQ.0) THEN

*     Inquire the GKS workstation identifier of the current zone.
        CALL SGS_ICURW(IWKID)

*     Encode workstation id as a character string.
        WRITE(UNIT=WKID, FMT='(I10)') IWKID

*     Open PGPLOT
        PGSTAT = PGBEG(0, WKID, 1, 1)
        IF (PGSTAT.EQ.1) THEN

*     Define axis limits.
           CALL PGENV(XLO, XHI, YLO, YHI, 0, 0)

*     Plot the data.
           CALL PGPT(N, X, Y, 2)

*     Close down PGPLOT.
           CALL PGEND
        END IF
     END IF
     END
\end{terminalv}
\end{small}

Because other plotting packages may have plotted on the same physical device,
there are some restrictions when using PGPLOT in this way:

\begin{itemize}
\item \xref{\texttt{PGVSIZ}}{sun15}{PGVSIZ} cannot be used.

\item \xref{\texttt{PGPAGE}}{sun15}{PGPAGE} will never clear the
screen.  If the display surface has been divided into sub-pictures
\texttt{PGPAGE} will move to the next sub-picture in the usual way.

\item On devices with fixed colour tables, the default PGPLOT colour
table will not be set up by \xref{\texttt{PGBEG}}{sun15}{PGBEG} unless the
display surface is empty.

\end{itemize}

\section{\xlabel{other_differences}Other Differences}
\label{other_differences}

In general the two versions produce nearly identical results when run on
the same device but the following differences should be noted:

\begin{itemize}
\item The GKS version of \xref{\texttt{PGPAGE}}{sun15}{PGPAGE} cannot be
used to make the display surface larger than the default size provided by
\xref{\texttt{PGBEG}}{sun15}{PGBEG}.
\item The GKS version of \texttt{PGBEG} clears the display surface immediately
instead of waiting until the first vector is plotted.
\end{itemize}

\section{\xlabel{support}Support}
\label{support}

PGPLOT is Starlink supported software and bugs should be reported through the
usual channels and not by contacting Dr Pearson directly. Problems with the GKS
specific code will be dealt with by Starlink but all changes to the code which
is common to the two versions of PGPLOT must be made in collaboration with Dr
Pearson.

The mixing of calls to PGPLOT and GKS routines is not supported except as
described in \slhyperref{this section}{section~}{}{viewport}, and neither
version supports the calling of GRPCKG routines directly. Existing programs
that call GRPCKG should be re-written to call the equivalent PGPLOT routines.




% ? End of main text
\end{document}
