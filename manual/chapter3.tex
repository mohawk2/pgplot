% PGPLOT manual, 16-MAY-1989, T. J. Pearson
%------------------------------------------------------------------------
% Copyright (c) 1983, 1984, 1985, 1986, 1987, 1988, 1989 by
% California Institute of Technology.
% All rights reserved.
%------------------------------------------------------------------------

\beginchapter{3}{WINDOWS AND VIEWPORTS}

\beginsection  Introduction

This chapter is concerned with positioning a graph on the screen or
hardcopy page, and controlling its scale.  In simple applications, the
position and scale of the graph are controlled more-or-less
automatically by the routine |PGENV|, but in order to obtain complete
control of positioning and scaling, it is necessary to understand the
concepts of the {\it View Surface}, the {\it Window}, and the {\it
Viewport}, and two coordinate systems: {\it World Coordinates\/} and {\it
Device Coordinates}. 

A simple PGPLOT picture might be a two-dimensional graph showing the
dependence of one variable on another. A typical graph has data points,
represented by error bars or special markers such as dots or diamonds,
possibly connected by lines, or perhaps plotted on the same scale as a
theoretical model drawn as a smooth curve. The graph must be labeled
with two axes to indicate the coordinate scales. 

The programmer must describe to PGPLOT the various elements of
the graph in terms of rectangular Cartesian coordinates.  The only
limitation on the coordinates is that they be represented as
floating-point (|REAL*4|) numbers; otherwise we are totally free to
choose the meaning of the coordinates. For example, in a graph showing
the temporal variation of a radio source, the abscissa ($x$-coordinate)
might be Epoch (in years) and the ordinate ($y$-coordinate) Flux Density
(in Jy). 

In accordance with common practice in graphics programming, these
coordinates, chosen by the programmer, are termed {\it world
coordinates\/}. PGPLOT maps a selected rectangular region of the
world-coordinate space (termed the {\it window\/}) onto a specified
rectangle (termed the {\it viewport\/}) on the {\it view surface\/} (the
screen of an interactive display or a sheet of paper on a hardcopy
plotter).  The program must make calls to PGPLOT routines to define both
the window and the viewport. For complete descriptions of the routines 
and their arguments, refer to Appendix A.


\beginsection Selecting a View Surface

The first thing a graphics program must do is to tell PGPLOT what
device it is going to use. This is done by calling routine |PGBEGIN|.
For example, to create a plot file for the Versatec printer: 
\begintt
CALL PGBEGIN (0, 'PLOTFILE.LIS/VERSATEC', 1, 1)
\endtt
Equally important, when all plotting has been completed, it is necessary 
to call |PGEND| to flush any pending plot requests: 
\begintt
CALL PGEND
\endtt
Note that only one device can be used at a time.  If |PGBEGIN| is called 
while a plot is in progress, the old plot is closed and a new one is 
begun.

After calling |PGBEGIN| the program has access to a {\it view surface}.  For
interactive devices, this is the full screen of the device. For hardcopy
devices, it is a standard page, usually $10 (x) \times 8.5 (y)$ inches on
a device used in ``landscape'' mode (\eg, device types |/VE| and |/QMS|),
or $8.5 (x) \times 10 (y)$ inches a device used in ``portrait'' mode 
(\eg, device types |/VV| and |/VQMS|).

On some devices, it is possible to plot on a larger piece of paper than 
the standard page; see the description of routine |PGPAPER|, which must 
be called immediately after |PGBEGIN| to change the size of the view 
surface. The different devices differ not only in the size of the
view surface, but also in its {\it aspect ratio\/} (height/width).
|PGPAPER| can be called to ensure that a plot has the same aspect ratio
no matter what device it is plotted on. 

After completing a graph, it is possible to advance to a new page to start a
new graph (without closing the plot file) by calling |PGPAGE|:
\begintt
CALL PGPAGE
\endtt
This clears the screen on interactive devices, or gives a new piece of
paper on hardcopy devices. It does not change the viewport or window.

The last two arguments of |PGBEGIN| (|NX| and |NY|) can be used to
sub-divide the view surface into smaller pieces called sub-pages, each
of which can then be used separately. The view-surface is divided into
$NX$ (horizontally) by $NY$ (vertically) sub-pages. When the view
surface has been subdivided in this way, |PGPAGE| moves the plotter
to the next sub-page, and only clears the screen or loads a new piece of
paper if there are no sub-pages left on the current page.

In addition to selecting the view surface, |PGBEGIN| also defines a 
default viewport and window.  It is good practice, however, to define 
the viewport and window explicitly as described below.


\beginsection Defining the Viewport

A {\it viewport} is a rectangular portion of the plotting surface onto which
the graph is mapped. PGPLOT has a default viewport which is centered on
the plotting surface and leaves sufficient space around it for
annotation.  The application program can redefine the viewport by
calling routine |PGVPORT| or |PGVSIZE|. 

|PGVPORT| defines the viewport in a device-independent manner, using a
coordinate system whose coordinates run from 0 to 1 in both $x$
and $y$. This coordinate system is called {\it normalized device
coordinate space\/}. For example, if we wish to divide the view surface
into four quadrants and map a different plot onto each quadrant, we can
define a new viewport before starting each plot.  |PGVPORT| has the
format: 
\begintt
CALL PGVPORT (XMIN, XMAX, YMIN, YMAX)
\endtt
For example, to map the viewport onto the upper left quadrant of the
view surface:
\begintt
CALL PGVPORT (0.0, 0.5, 0.5, 1.0)
\endtt
(Note that this does not leave room around the edge of the viewport for
annotation.)

|PGVSIZE| defines the viewport in absolute coordinates (inches); it
should only be used when it is known how big the view surface is and a
definite plot scale is required.  The arguments are the same as for
|PGVPORT|, but measured in inches from the bottom left corner of the
view surface. For example: 
\begintt
CALL PGVSIZE (1.5, 9.5, 1.5, 6.5)
\endtt
defines a rectangular viewport 8 by 5 inches, offset 1.5 inches from the
bottom and left edges of the view surface.

|PGVSTAND| defines a standard viewport, the size of which depends
on the particular device being used, and on the current character size
(it uses the whole view surface excluding a margin of four character 
heights all around):
\begintt
CALL PGVSTAND
\endtt
This is the default viewport set up by |PGBEGIN|.

Note that the viewport must be defined {\it before\/} calling any routines
that would actually generate a display. The viewport may, however, be
changed at any time: this will affect the appearance of objects drawn
later in the program. 


\beginsection Defining the Window

The program defines the {\it window} by calling routine |PGWINDOW|, whose
arguments specify the world-coordinate limits of the window
along each coordinate axis. \eg: 
\begintt
CALL PGWINDOW (1975.0, 1984.0, 5.0, 20.0)
\endtt
specifies that the $x$-axis (epoch) is going to run (left to right) from
1975 to 1984, and the $y$-axis (flux density) is going to run (bottom to
top) from 5 to 20~Jy. Note that the arguments are floating-point numbers
(Fortran |REAL| variables or constants), and require decimal points. If
the order of either the $x$ pair or the $y$ pair is reversed, the
corresponding axis will point in the opposite sense, \ie, right to left
for $x$ or top to bottom for $y$. PGPLOT uses the window specification
to construct a mapping that causes the image of the window to coincide
with the viewport on the view surface. Furthermore, PGPLOT ``clips''
lines so that only those portions of objects that lie within the window
are displayed on the view surface. 

Like the viewport, the window must be defined before drawing any
objects. The window can be defined either before or after the viewport:
the effect will be the same.  The default window, set up by |PGBEGIN|,
has $x$~limits 0.0--1.0 and $y$~limits 0.0--1.0.

If the ratio of the sides of the window does not equal the ratio of 
the sides of the viewport, the mapping of the world coordinates onto 
the view surface results in an image whose shape is compressed in either 
$x$ or $y$.  One way to avoid this compression is to carefully choose
the viewport to have the same aspect ratio as the window.  Routine
|PGWNAD| can do this: it defines the window and simultaneously adjusts
the viewport to have the same aspect ratio as the window.  The new 
viewport is the largest that can fit inside the old one, and is centered 
in the old one.


\beginsection Annotating the Viewport

For a simple graph, it is usually necessary to draw a frame around
the viewport and label the frame with tick marks and numeric labels.
This can be done with the routine |PGBOX|.  For
our sample graph, the call might be:
\begintt
CALL PGBOX ('BCTN', 0.0, 0, 'BCNST', 0.0, 0)
\endtt
Another routine, |PGLABEL|, provides text labels for the bottom, left
hand side, and top of the viewport:
\begintt
CALL PGLABEL ('Epoch', 'Flux Density (Jy)',
              'Variation of 3C345 at 10.7 GHz')
\endtt
The first two arguments provide explanations for the two axes; the
third provides a title for the whole plot. Note that unlike all the
other plotting routines, the lines and characters drawn by |PGBOX| and
|PGLABEL| are not clipped at the boundaries of the window.  |PGLABEL|
actually calls a more general routine, |PGMTEXT|, which
can be used for plotting labels at any point relative to the viewport.

The amount of space needed outside the viewport for annotation
depends on the exact options specified in |PGBOX|; usually four
character heights will be sufficient, and this is the amount allowed
when the standard viewport (created by |PGVSTAND|) is used. The character
height can be changed by using routine |PGSCH|.


\beginsection Routine PGENV

Having to specify calls to |PGPAGE|, |PGVPORT|, |PGWINDOW|, and
|PGBOX| is excessively cumbersome for drawing simple graphs. Routine
|PGENV| (for PGplot ENVironment) combines all four of these in one
subroutine, using the standard viewport, and a limited set of the
capabilities of |PGBOX|.  For example, the graph described above could
be initiated by the following call: 
\begintt
CALL PGENV (1975.0, 1984.0, 5.0, 20.0, 0, 0)
\endtt
which is equivalent to the following series of calls:
\begintt
CALL PGPAGE
CALL PGVSTAND
CALL PGWINDOW (1975.0, 1984.0, 5.0, 20.0)
CALL PGBOX ('BCNST', 0.0, 0, 'BCNST', 0.0, 0)
\endtt
|PGENV| uses the standard viewport.  The first four arguments define the
world-coordinate limits of the window. The fifth argument can be 0 or 1; 
it is 1, |PGENV| calls |PGWNAD| instead of |PGWINDOW| so that the plot
has equal scales in $x$ and $y$. The sixth argument controls the amount
of annotation.

\endchapter
