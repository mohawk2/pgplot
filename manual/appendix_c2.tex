% PGPLOT manual, T. J. Pearson    29-MAY-1989
%------------------------------------------------------------------------
% Copyright (c) 1983, 1984, 1985, 1986, 1987, 1988, 1989 by
% California Institute of Technology.
% All rights reserved.
%------------------------------------------------------------------------

\beginappendix{C2}{INSTALLATION INSTRUCTIONS (UNIX)}

\beginsection Introduction

All UNIX systems are different, so the installation procedure described 
here may not work on your system, and some editing and debugging may be 
required.  Please read all the installation instructions before
attempting to install PGPLOT. I first describe the installation
procedure which I have used on our systems (Convex and Sun); I hope it
will work on other Berkeley-derived UNIX systems. If you have trouble
with the installation, you should read Section~C.3, which is
intended to provide sufficient background information for you to adapt
PGPLOT to a different system. 

The UNIX version of PGPLOT is distributed in source form on a `tar' tape
(usually a 9-track, 1600-bpi tape).

\beginsection Basic Installation

The following installation procedure has been run on a Convex C-1
running Version 6.2 or 7.0 of the Convex operating system with Fortran
compiler FC Version 4.1 or 5.0, and on Sun-3 and Sun-4 workstations
running SunOS Release 4.0 with Sun Fortran Release 1.1. It does {\it
not\/} work with earlier versions of the Convex Fortran compiler, the
Sun operating system, or the Sun Fortran compiler. 

\bigskip\noindent
1. Copy the files from the distribution tape to your disk. The files
are organized in three directories: |pgplot|, |pgplot/examples|, and 
|pgplot/fonts|. To copy the files from the `tar' tape, you should first
change your default directory to the {\it parent\/} directory of 
|pgplot|, and use a command like the following:
\begintt
tar xv pgplot
\endtt
This will create a |pgplot| directory in your current directory and 
install the PGPLOT files in it. (If you already have a |pgplot| 
directory, remove it first.) If you are not using `tar' but are copying 
the files from another system, make sure that you put the files in the
correct directories.

\bigskip\noindent
2. Compile the PGPLOT subroutines and create the PGPLOT object library.
Change your default directory to PGPLOT and then use `make' to do the 
compilation. The distribution tape contains two `makefiles', one for
Convex (|Makefile.CONVEX|) and one for Sun (|Makefile.SUN|). You can
rename the appropriate file to |Makefile|, or use the |-f| option of 
`make':
\begintt
cd pgplot
make -f Makefile.SUN
\endtt
If all goes well, you should get no error messages from the compilation.
(The SUN Fortran compiler issues an incorrect warning message that |MOD|
is an unused variable when compiling |grre04| and |grte04|; you can 
ignore this message.)

\bigskip\noindent
3. Build the binary font file.  The font file (read at run time by 
PGPLOT programs) is distributed in an ASCII text form in file 
|grfont.txt| and must be converted to binary form before use. This is
done using the `makefile' in the |fonts| directory; again there are
separate versions for Convex and Sun: 
\begintt
cd fonts
make -f Makefile.SUN grfont.dat
\endtt
This compiles a program |pgpack| to do the conversion and then runs it.
It should display a message like the following:
\begintt
Characters defined:   996
Array cells used:   26732
\endtt
You now need to define an environment variable that defines the location 
of the binary font file, e.g.:
\begintt
setenv PGPLOT_FONT /usr/name/pgplot/fonts/grfont.dat
\endtt
Substitute the appropriate path for |/usr/name/pgplot|; or while you are 
in the |fonts| directory, type
\begintt
setenv  PGPLOT_FONT `pwd`/grfont.dat
\endtt
The |fonts| directory also contains utility programs |pgunpack| and
|pgdchar| that you will need only if you plan to modify the font file (not
recommended). 

\bigskip\noindent
4. Compile the test and demonstration programs in the |examples| 
directory. Again there there are separate `makefiles' for Convex 
and Sun:
\begintt
cd ../examples
make -f Makefile.SUN
\endtt
At this point, you may run out of disk space (if you haven't done 
already) as the executable programs are rather large. If necessary, you
can compile the programs one at a time, e.g.:
\begintt
make -f Makefile.SUN pgdemo1
\endtt
to compile program |pgdemo1|. The two demonstration programs
|pgdemo1| and |pgdemo2| each generate several screens (pages) 
illustrating the use of PGPLOT subroutines. There is also a general test 
program, |pgex17|, described in Appendix~E, and two test programs for 
the cursor routines, |pgex15| and |pgex18|.

\bigskip\noindent
5. Run at least one of the test programs, e.g. |pgdemo1|.
\begintt
pgdemo1
\endtt
The program will prompt for a device specification. Type a question mark 
|?| to get a list of available device handlers. Read the PGPLOT manual
for details of device specifications, and the remainder of this 
Appendix for notes on the available device handlers. Run the program 
once for each device that you wish to test. I hope that all will go 
well; if not, read the next section. The commonest problem is that
you get graphs with no text. This implies that the program cannot read 
the font file (you should also get a message to this effect). Check the
definition of |PGPLOT_FONT| (see above).

\bigskip\noindent
6.  If everything so far looks all right, you can move the files to 
their final destinations (this may require superuser privilege and the 
cooperation of your system manager). You need to keep two files: the
object library |pgplot/libpgplot.a| and the binary font file
|pgplot/fonts/grfont.dat|. Everything else can be deleted if you want to
conserve disk space; it can always be recovered from the distribution
tape.  The object library should (ideally) be moved to the directory
where the loader expects to find libraries; usually |/usr/lib|: 
\begintt
cd pgplot
cp libpgplot.a /usr/lib
\endtt
You can then use |-lpgplot| with |ld| or your Fortran compiler to
link a program with the PGPLOT library.  The location of the font file 
is arbitrary, but as distributed, PGPLOT expects to find it in 
|/usr/local/lib/grfont.dat|:
\begintt
cd fonts
cp grfont.dat /usr/local/lib
\endtt
If you want to put it in some other location, you can {\it either\/} 
define |PGPLOT_FONT| every time you use a PGPLOT program (see above), 
{\it or\/} modify the default location by editing file |pgplot/grsy00.f|
and recompiling (see step 2). (Note that if you modify any PGPLOT
routine, you will have to remodify it each time you install a new
version of PGPLOT.) 

\bigskip\noindent
7. Finally, you should attempt to compile and link a PGPLOT program
of your own. On Convex:
\begintt
fc -o program program.f -lpgplot
\endtt
On Sun:
\begintt
f77 -o program program.f -lpgplot -lsuntool -lcgi77 -lcgi\
    -lsunwindow -lpixrect -lm
\endtt
Note that the Sun version requires all the above libraries to be 
included, in the order indicated (assuming that the /SUNVIEW and /CGI
device handlers are included, as they are in the distribution tape).

\beginsection Advanced Installation

This section addresses problems that you may encounter trying to install 
PGPLOT on a different UNIX system, and indicates what options are open 
to you to solve them.

Examine the `makefiles' in the |pgplot| directory and make sure that you 
understand what they do. In particular, you may need to change the 
definitions of the Fortran compiler and compilation switches.

The subroutines are organized into several groups in the `makefile':
\smallskip
\item{1.} PG routines.  These are the top level PGPLOT routines that 
can be called by application programs, plus a few internal routines.
They are identical to the VMS version, and are written in standard
Fortran-77 (I believe) with two exceptions; the |INCLUDE| statement and
the use of subroutines names longer than 6~characters. The former can be
avoided by use of a preprocessor or by hand-editing the code to replace
each |INCLUDE| statement with the text of the included file; the latter
is fundamental to PGPLOT, but you might try truncating names to 6
characters. 

\item{2.} GR routines.  These are internal routines. They are also 
supposed to be standard Fortran-77, but there are a few problems.
As in the PG routines, some subroutine names are longer than 
6~characters, and the |INCLUDE| statement is used. A large array
|BUFFER| used in |grsy00.f| and |grsyxd.f| is declared |INTEGER*2| to save 
space both internally and in the binary font file (the array is just
a copy of the disk file); if you change it to |INTEGER|, you must make
the same change in the |pgpack| program in the |fonts| directory.
Routine |grgrps.f| uses a hexadecimal format code (|Z2.2|) which is
non-standard.

\item{3.} Drivers.  Each device handler is a subroutine with a name
like |xxdriv| (for device |xx|). The `makefile' lists the additional
subroutines required for each handler. The |DRIVERS| macro lists all
the device handlers to be included in the final library. You customize 
PGPLOT by changing the list assigned to |DRIVERS|.  The device handlers
are not written in standard Fortran-77, although I have tried to adhere 
to the standard where possible.  Please let me know what problems you 
have.  Some of the handlers call on subroutines written in C rather than 
Fortran. The conventions for writing a C subroutine to be called from a 
Fortran program are different on different systems (it may not even be 
possible). Thus you may need to do a lot of work to get each handler
going. But note that you don't need to have {\it all\/} the handlers
running in order to test PGPLOT.  It is sufficient to include just the
null device handler, which is written in standard Fortran-77. (See below 
for notes on each handler.)

\item{4.} Dispatch routine. The routine |grexec| must be modified to 
include the list of device handlers you wish to include in PGPLOT.
Two different versions are provided on the distribution tape; one for 
Convex and one for Sun.
It is a simple multi-way |GOTO| (|CASE| statement), one branch for
each handler. Eventually this routine will be assembled automatically 
during the installation procedure, but at present you must edit it by 
hand.

\item{5.} System routines. These routines, called from all levels of 
PGPLOT, encapsulate various operating-system dependent functions.
If you have trouble with these routines, you will need to write new
versions. I hope that the internal comments are sufficient to explain 
what each routine is supposed to do.  The version distributed calls
on a number of Fortran library functions which are normally part of 
Berkeley-UNIX but which may not be present in other varieties (e.g.,
|getenv|, |getlog|, |hostnm|).

\item{6.} Obsolete routines.  This group includes a number of routines
that are no longer an official part of PGPLOT, but they may be called
by some old PGPLOT application programs.  If you have such programs,
I recommend you rewrite them, but if necessary you can compile an 
additional library containing these routines by the command
|make -f Makefile.SUN libpgobs.a|~.

\beginsection Device Handlers

The UNIX version of PGPLOT is distributed with a much smaller set of 
device handlers than the VMS version.  If you modify any of the device 
handlers or write new ones, please (a) try to keep them as portable as 
possible, and (b) send copies to me so that I can include them in future 
distributions. See Appendix~E for instructions on writing a device 
handler.

\beginsub{Null device}
The null device handler is |nudriv|. Use device specification |/NULL|. 
Output sent to the null device is discarded; i.e., PGPLOT produces no
output.  This device handler is fully portable. 

\beginsub{SunView} (Sun only)
File |svdriv| is a first attempt at a handler for Sun workstations running
SunView, from Brian M. Sutin (sutin@astro.umd.edu). It is written in C. 
Programs which use it must be linked with several libraries (in order): 
\begintt
-lsuntool -lsunwindow -lpixrect
\endtt
Use device specification |/SUNVIEW|. Please let me know if you have any 
problems with this handler, or make improvements to it.

\beginsub{Sun-CGI} (Sun only)
File |cgdriv| is another handler for Sun workstations running
SunView, from Allyn Tennant. It only works with color workstations, but
could probably be easily modified for monochrome workstations. It calls
on the Sun CGI routines, and hence programs which use it must be linked
with several libraries (in order): 
\begintt
-lcgi77 -lcgi -lsunwindow -lpixrect -lm
\endtt
Use device specification |/CGI|. PGPLOT creates a window on the screen,
or uses the whole screen if you are not running SunView. There is no
cursor available with this handler. The window is closed when your 
program exits, so it is not possible to overlay one plot on top of 
another by using the undocumented |/APPEND| qualifier.

\beginsub{IVAS} (Convex only)
The handler for the IIS IVAS image display (|ivdriv|, device 
specification |/IVAS|) calls C routines to do most of the work.  It is
self-contained (no other libraries are required), but will require a
compatible version of the Convex device driver for the interface board. 

\beginsub{Imagraph} (Convex only)
This is a handler for the Imagraph image display (|tvdriv|, device
specification |/IMAGRAPH|). The board is an AGC SERIES VME-1280-10 board
(IMAgraph Corporation, 800 West Cummings Park, Woburn, Massachusetts
01801), and utilizes a Hitachi HD63484 Advanced CRT Controller (ACRTC)
as graphics engine. It is configured with $1024 \times 1024$ 8-bit pixels
(image frame memory) plus a 2-bit pixel overlay. The device handler is
self-contained (no other libraries are required), but it will require
the Caltech Convex device driver for the interface board
(contact Judy Cohen for more information). 

\beginsub{Tektronix terminals and emulators}
The basic Tektronix-4010 handler is |tedriv| (|/TEK|). Extensions for
use with Retrographics terminals and Graph\-On terminals are provided in
files |redriv| (|/RETRO|) and |gfdriv| (|/GF|), respectively. The code for
these handlers is fairly standard, with the exception of C routines in
|gr_term_io.c| for writing to the terminal. As control characters must
be output, the terminal is put in `cbreak' mode during output.  The
|tedriv| handler works with the Sun Tektronix emulator |tektool|. The 
cursor works with these device handlers on the Convex but not on Suns:
I would appreciate suggestions for fixing this problem.

\beginsub{PostScript printers}
The PostScript handlers (|psdriv| for landscape mode, |vpdriv| for 
portrait mode) adhere fairly closely to standard Fortran-77, and so 
should be easily ported to different systems.
The files that they create do not display correctly in the Sun
|psview| PostScript preview program owing to what I assume is a bug in
Sun's implementation of PostScript. The PostScript handlers can 
optionally insert a control-D (end-of-file) character at the beginning
and end of the PostScript file, which may be needed by some PostScript 
printers. To enable this option, define the environment variable
|PGPLOT_PS_EOF| (with any value).

\beginsub{REGIS terminals (e.g., DEC VT125)}
The REGIS device handler is in file |vtdriv|; specify a device type
of |/VT125|.  It is fairly close to standard Fortran-77. The default
output device is the user's terminal, but a disk file can also be 
specified. The display should be satisfactory on VT125 terminals; later
DEC terminals (like VT240, VT340) do not have separate memories for text
and graphics, and this leads to problems with programs that interleave
text and graphical output. At present, the cursor is not implemented. 

\beginsub{Other devices} (Convex only)
The basic installation procedure installs some additional drivers in the
Convex version of PGPLOT, but these have not been tested in other
versions. Imagen printers: |imdriv|; Printronix printers: |pxdriv|; QMS
printers: |qmdriv| and |vqdriv|; Versatec printers: |vedriv| and
|vvdriv|.

\beginsection Special Notes: Sun

The Sun Fortran compiler treats the `backslash' character (|\|) 
as a special escape character in literal character strings. This means 
that any programs that try to make use of the PGPLOT escape-character 
mechanism, which also uses backslash, will have problems.  Until Sun 
provides a way to disable this annoying behavior of the compiler, the
solution is to replace each occurrence of |\| by |\\|. This does not
cause problems within the PGPLOT library (there are no literal |\|
characters), but it does affect the example programs and many
application programs.  The `makefile' in the |examples| directory passes
the example programs through the Sun filter |f77cvt| which performs this
substitution (among other things).

\beginsection Acknowledgments

I am grateful to Jon Danskin of Convex Computer Corporation for the 
initial port of PGPLOT to Convex-UNIX, Allyn Tennant (NASA Marshall) for 
providing the Sun-CGI device handler, Brian Sutin (University of 
Maryland) for the SunView device handler, Neil Killeen (University of 
Illinois) for comments on the Sun implementation, and Judy Cohen 
(Caltech) for providing time on Sun-3 and Sun-4 workstations. 

\bigskip
\rightline{T. J. Pearson}
\rightline{29 May 1989}

\endchapter
