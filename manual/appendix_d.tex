% PGPLOT manual,  5-JUL-1989, T. J. Pearson
%------------------------------------------------------------------------
% Copyright (c) 1983, 1984, 1985, 1986, 1987, 1988, 1989 by
% California Institute of Technology.
% All rights reserved.
%------------------------------------------------------------------------

\beginappendix{D}{SUPPORTED DEVICES}

\beginsection Introduction

This Appendix presents device-specific information for some of the
supported devices. Table \the\chapnum.1 shows the devices for which
device handlers are available, together with the names by which they are
known to PGPLOT. The names of the device types can be abbreviated so
long as there is no ambiguity; in most cases, this means the first two
letters are sufficient. Each installation of PGPLOT is configured with
the devices appropriate for that installation, so not every device is
available in every installation of PGPLOT. Some devices are available
under VMS only, and others are available under Unix only. 

\pageinsert
  \def\head#1{\noalign{\medskip\centerline{\bf #1}\smallskip}}
  \centerline{\bf Table \the\chapnum.1\quad Available Devices}
\halign{\indent#\hfil&  \quad\tt#\hfil\cr
\head{Terminals}
GraphOn GO-230 terminal                         &/GF \cr
Tektronix 4006/4010 storage-tube terminal       &/TEK4010 \cr
Retrographics modified VT100 terminal (VT640)   &/RETRO \cr
DEC VT125, VT240, or VT340 terminal (REGIS)     &/VT125 \cr
Tektronix 4100-series color terminal            &/TK4100 \cr
ZSTEM 240/4014 terminal emulators (IBM PC)      &/ZSTEM \cr
\head{High-resolution Display Devices}
DeAnza (Gould 8500 low resolution)              &/DEanza \cr
Grinnell GMR-270 Image Display System           &/GRINNELL \cr
Digisolve Ikon Pixel Engine                     &/IKon \cr
Liacom Graphic Video Display (GVD-02)           &/LIacom \cr
Peritek Corp. VCH-Q frame-buffer video          &/PERITEK \cr
Peritek Corp. VCK-Q frame-buffer video          &/PK \cr
Sigma Args, 7000 series                         &/ARgs \cr
Sigma, T5670 terminal                           &/GOC \cr
Tektronix 4014 (12 bit addressing)              &/TFILE \cr
VAX/VMS workstations                            &/WS \cr
Sun workstations (SunView mode)                 &/SUNVIEW \cr
Sun workstations (using CGI routines)           &/CGI \cr
IIS IVAS image display                          &/IVAS \cr
Imagraph image display                          &/IMA \cr
\head{Pen Plotters}
Gould (now Bryans) Colourwriter 6320            &/CW6320 \cr
HPGL Driver (tested on HP7475A plotter)         &/HPGL \cr
HPGL Driver (portrait mode)                     &/VHPG \cr
Hewlett Packard 7221 pen plotter                &/HP7221 \cr
Houston Instruments HIDMP pen plotter           &/HIDMP \cr
Bruning (fmr Nicolet) Zeta 8 Digital Plotter    &/ZEta \cr
\head{Laser Printers}
Hewlett Packard LaserJet, LaserJet+,II          &/LJ \cr
PostScript device (landscape mode)              &/PS \cr
PostScript device (portrait mode)               &/VPS \cr
QUIC devices (QMS 800 etc) (landscape mode)     &/QMS \cr
QUIC devices (QMS 800 etc) (portrait mode)      &/VQMS \cr
Canon LBP-8/A2 Laser printer (landscape mode)   &/BCanon \cr
Canon LBP-8/A2 Laser printer (landscape mode)   &/CAnon \cr
Canon LBP-8/A2 Laser printer (portrait mode)    &/VCanon \cr
Impress (Imagen) device (landscape mode)        &/IMPRESS \cr
Impress (Imagen) device (portrait mode)         &/VIMPRESS \cr
}
\vfill
\endinsert

\pageinsert
  \def\head#1{\noalign{\medskip\centerline{\bf #1}\smallskip}}
  \centerline{\bf Table \the\chapnum.1\quad (continued)}
\halign{\indent#\hfil&  \quad\tt#\hfil\cr
\head{Dot-matrix Printers}
EPSON FX100 printer                             &/EPSON \cr
EXCL (C.Itoh Megaserve) printer                 &/EXCL \cr
LA50 and other DEC sixel printers               &/LA50 \cr
Printronix P300 or P600 printer                 &/PRINTRONIX \cr
Toshiba ``3-in-on''" printer, model P351        &/TOSHIBA \cr
Versatec V80 printer (landscape mode)           &/VERSATEC \cr
Versatec V80 printer (portrait mode)            &/VVERSATEC \cr
\head{Other}
Null device (no graphical output)               &/NULL \cr
Metafile                                        &/FILE \cr
}
\vfill
\endinsert

The description of each device is organized under the following headings:

\proclaim Supported device: description of device.

\proclaim Device type code: the code to be used (following the /) in a
PGPLOT device specification; usually this can be abbreviated to two 
letters.  Some devices can be used in two modes: {\it landscape\/} (long 
axis horizontal) and {\it portrait\/} (long axis vertical). Different 
device type codes are used for the two modes.

\proclaim Default file or device name: the file or device name that is 
used by default if none is included in the device specification. (For 
example, if the device specification is given as |/VERS|, it is expanded
to |PGPLOT.VEPLOT/VERS|; if it is given as |UV/VERS|, it is expanded to
|UV.VEPLOT/VERS|.) 

\proclaim Default view surface dimensions:
Most hardcopy devices print on $11\times8.5$-inch paper, and the 
standard usable ``view surface'' is $10.5\times8.0$ inches. The 
dimensions are meaningless for most CRT devices where the image size 
depends on the size of the monitor.

\proclaim Resolution:
The nominal resolution of the device in pixels/inch. The exact 
resolution can be ill-defined; \eg, on a pen plotter, it depends on the
size of the pen nib.

\proclaim Color capability:
What color indices are available?  Can the representation of each 
color index be changed?  Do changes to the representation affect 
primitive elements that have already been drawn?

\proclaim Input capability:
Instructions for the use of the graphics cursor, if one is available.

\proclaim File format:
Description of the file in which the graphic commands are stored. For
some devices, the commands are sent directly to the device and no 
intermediate file is required.

\proclaim Obtaining hardcopy:
When an intermediate file is used, this section gives instructions for 
sending the graphic commands from the file to the device.
{\it It is important to check the correct procedure with your
System Manager before attempting to generate a hardcopy plot. 
Attempting to print or plot a file on the wrong sort of device will at
best waste a large quantity of paper, and may damage the device or crash
the VAX.}


\beginsection Versatec

\proclaim Supported device: Versatec V-80 printer/plotter (and compatible 
models).

\proclaim Device type code: |/VErsatec| (landscape mode), |/VVersatec|
(portrait mode).

\proclaim Default file name: |PGPLOT.VEPLOT| (landscape mode),
|PGPLOT.VVPLOT| (portrait mode).

\proclaim Default view surface dimensions:
10.5 inches horizontal $\times$ 8.0 inches vertical (landscape mode),
8.0 inches horizontal $\times$ 10.5 inches vertical (portrait mode).
These are nominal values; the actual scale may vary, particularly
in the dimension parallel to the paper motion.

\proclaim Resolution:
200 pixels/inch (both dimensions).

\proclaim Color capability:
Color indices 0 (erase) and 1 (black) are supported. Requests for other
color indices are converted to 1.  It is not possible to change color
representation.

\proclaim Input capability: None.

\proclaim File format:
The file contains variable length records (up to 265 bytes), one record
corresponding to one horizontal dot row of the plot. The file has record
attributes ``No Carriage Control.'' The first byte in each record is
control-D (hexadecimal 04: plot mode specifier). The remaining 8-bit
bytes each represent 8 dots, with the most significant bit representing
the left-most dot; 1 implies the corresponding dot is to be filled in.
Thus the maximum number of dots per line is $264 \times 8 = 2112$,
corresponding to 10.56 inches at 200 dots per inch. The vertical spacing
of dot rows is also 200 per inch.  Plot pages are separated by a record
containing a form-feed character only (1 byte, hexadecimal 0C). 
These files are intended for use with a special driver and printer
symbiont which recognize the control-D and send the remainder of the
record to the Versatec in plot mode instead of text mode. This is {\it
not\/} the format expected by the standard VMS driver provided by
Versatec. 

\proclaim Obtaining hardcopy:
Use the VMS |COPY| command to send the file to a suitable device or use
the VMS |PRINT| command if a suitable printer queue has been
established.
Examples: 
\begintt
$ COPY PGPLOT.VEPLOT LVA0:
$ PRINT/NOTIFY/QUEUE=VERSATEC PGPLOT.VEPLOT
\endtt
On DEIMOS: 
\begintt
$ PRINT PGPLOT.VEPLOT
\endtt
On PHOBOS: 
\begintt
$ VPRINT PGPLOT.VEPLOT
\endtt
On Starlink: 
\begintt
$ PRINT/PASSALL/QUEUE=SYS_VERSATEC PGPLOT.VEPLOT
\endtt


\beginsection PostScript printers

\proclaim Supported device: any printer that accepts the PostScript page 
description language, \eg, the LaserWriter (Apple Computer, Inc.).

\proclaim Device type code: |/PS| (landscape mode), |/VPS|
(portrait mode).

\proclaim Default file name: |PGPLOT.PSPLOT| (landscape mode),
|PGPLOT.VPPLOT| (portrait mode).

\proclaim Default view surface dimensions:
10.5 inches horizontal $\times$ 7.8 inches vertical (landscape mode),
7.8 inches horizontal $\times$ 10.5 inches vertical (portrait mode).

\proclaim Resolution:
Commands sent to the device use coordinate increments of 0.001~inch,
giving an ``apparent'' resolution of 1000 pixels/inch. The true
resolution is device-dependent; \eg, on an Apple LaserWriter it is 300
pixels/inch (in both dimensions). 

\proclaim Color capability:
Color indices 0 (erase), 1--13 (black), 14 (light grey), and 15 (dark grey)
are supported. Requests for other color indices are converted to 1.  
It is not possible to change color representation.

\proclaim Input capability: None.

\proclaim File format:
The file contains variable length records containing Post\-Script commands.
The commands use only printable ASCII characters, and the file can be
examined or modified with a text editor. 

\proclaim Obtaining hardcopy:
Use the VMS |COPY| command to send the file to a suitable device or use
the VMS |PRINT| command if a suitable printer queue has been
established.
On DEIMOS: 
\begintt
$ PRINT PGPLOT.PSPLOT/QUEUE=LW
\endtt

\proclaim References:
(1) Adobe Systems, Inc. {\it PostScript Language Reference Manual.}
Addison-Wesley, Reading, Massachusetts, 1985.
(2) Adobe Systems, Inc. {\it PostScript Language Tutorial and Cookbook.}
Addison-Wesley, Reading, Massachusetts, 1985.


\beginsection QMS Lasergrafix 

\proclaim Supported device: QMS Lasergrafix 800/1200 laser printers and 
other printers accepting ``QUIC'' commands.

\proclaim Device type code: |/QMs| (landscape mode), |/VQms|
(portrait mode).

\proclaim Default file name: |PGPLOT.QMPLOT| (landscape mode),
|PGPLOT.QEPLOT| (portrait mode).

\proclaim Default view surface dimensions:
10.5 inches horizontal $\times$ 8.0 inches vertical (landscape mode),
8.0 inches horizontal $\times$ 10.5 inches vertical (portrait mode).
These are nominal values; the actual scale may vary.

\proclaim Resolution:
300 pixels/inch (both dimensions). Commands sent to the device use
coordinate increments of 0.001~inch, giving an ``apparent'' resolution
of 1000 pixels/inch.

\proclaim Color capability:
Color indices 0 (erase) and 1 (black) are supported. Requests for other
color indices are converted to 1.  It is not possible to change color
representation.

\proclaim Input capability: None.

\proclaim File format:
The file contains variable length records containing QUIC commands.
The commands use only printable ASCII characters, and the file can be
examined or modified with a text editor. 

\proclaim Obtaining hardcopy:
Use the VMS |COPY| command to send the file to a suitable device or use
the VMS |PRINT| command if a suitable printer queue has been
established.
Examples: 
\begintt
$ COPY PGPLOT.QMPLOT TXA0:
$ PRINT/NOTIFY/QUEUE=LASER PGPLOT.QMPLOT
\endtt
On XHMEIA: 
\begintt
$ LASER PGPLOT.QMPLOT
\endtt
On CVAX:   
\begintt
$ PRINT/QUEUE=TXA0 PGPLOT.QMPLOT
\endtt


\beginsection Printronix

\proclaim Supported device: Printronix P300, P600 and equivalent
dot-matrix printer-plotters.

\proclaim Device type code: |/PRintronix| (landscape mode only).

\proclaim Default file name: |PGPLOT.PRPLOT|.

\proclaim Default view surface dimensions:
13.2 inches horizontal $\times$ 10.0 inches vertical, on standard 
$15\times11.0$-inch computer paper.  These are nominal
values, but the printer is usually quite accurately aligned.

\proclaim Resolution:
60 pixels/inch horizontal $\times$ 72 pixels/inch vertical.

\proclaim Color capability:
Color indices 0 (erase) and 1 (black) are supported. Requests for other
color indices are converted to 1.  It is not possible to change color
representation.

\proclaim Input capability: None.

\proclaim File format:
A Printronix file contains variable length records (up to 135 bytes),
one record corresponding to one horizontal dot row of the plot. The file
has record attributes ``No Carriage Control.'' The last three bytes in
each record are control-E (hexadecimal 05: plot mode specifier),
carriage-return, line-feed. The remaining bytes each use the 6
lower-order bits to represent 6 dots, with the least significant bit
representing the left-most dot; 1 implies the corresponding dot is to be
filled in. The top two bits are always 1. Thus the maximum number of dots
per line is $132 \times 6 = 792$, corresponding to 13.2 inches at 60
dots per inch. If a
plot covers more than one page, no form-feed codes are inserted; rather
a sufficient number of blank plot lines are inserted to advance to the
top of the next page (792 lines per 11-inch page). If these files are to
be printed on a Printronix printer using the standard VAX/VMS
line-printer driver (LPDRIVER or LCDRIVER), the system manager must set
the characteristics of the printer and associated queue correctly. It is 
important to ensure that (1) all 8-bit characters, including
non-printable control characters, are passed to the printer, (2) lines
of 135 bytes or less are not truncated, and (3) no extra formatting 
commands (e.g., form-feeds) are sent to the printer. The following
setup is used on Phobos:
\begintt
$ SET PRINTER/LOWER/CR/PRINTALL LPA0:
$ SET DEVICE/SPOOLED=(SYS$PRINT,SYS2:)	LPA0:
$ DEFINE/FORM/NOTRUNC/NOWRAP DEFAULT 0
$ INITIALIZE/QUEUE/START/ON=LPA0: SYS$PRINT -
      /DEFAULT=(FLAG=ONE)/PROT=(W:RW)/SCHED=NOSIZE
\endtt

\proclaim Obtaining hardcopy:
On VAX/VMS machines, Printronix plot files can be printed with a
standard |PRINT| command, but the |/PASSALL| qualifier {\it must\/} be
included. (If the printer is setup as in the example above, |/NOFEED|
can be substituted for |/PASSALL|, but this is not recommended.) 
Examples: 
\begintt
$ PRINT/NOTIFY/PASSALL PGPLOT.PRPLOT
\endtt
On Phobos: 
\begintt
$ PRINT/PASSALL PGPLOT.PRPLOT
\endtt
On Deimos: 
\begintt
$ PPRINT/PASSALL PGPLOT.PRPLOT
\endtt
On Starlink: 
\begintt
$ PRINT/PASSALL/QUEUE=SYS_PRINTRONIX PGPLOT.PRPLOT
\endtt
On Ikaros (Convex-Unix): 
\begintt
lpr -l PGPLOT.PRPLOT
\endtt


\beginsection VT125 (DEC REGIS terminals)

\proclaim Supported device: 
Digital Equipment Corporation VT125, VT240, or VT241 terminal; other
REGIS devices may also work. 

\proclaim Device type code: |/VT125|.

\proclaim Default file name (VMS): |TT:PGPLOT.VTPLOT|. This usually means the
terminal you are logged in to (logical name |TT|), but the plot can be
sent to another terminal by giving the device name, \eg, |TTC0:/VT|, or
it can be saved in a file by specifying a file name, \eg, 
|CITSCR:[TJP]XPLOT/VT| (in this case a disk name must be included as 
part of the file name).

\proclaim Default file name (Unix): |/dev/tty|, the
terminal you are logged in to. the plot can be
sent to another terminal by giving the device name, or
it can be saved in a file by specifying a file name.

\proclaim Default view surface dimensions: Depends on monitor.

\proclaim Resolution: The default view surface is 768 (horizontal)$\times$
460 (vertical) pixels.  On most Regis devices, the resolution is 
degraded in the vertical direction giving only 230 distinguishable 
raster lines. (There are actually 240 raster lines, but 10 are reserved 
for a line of text.)

\proclaim Color capability:  Color indices 0--3 are supported.
By default, color index 0 is black (the background color). Color indices
1--3 are white, red, and green on color monitors, or white, dark grey, and
light grey on monochrome monitors.  The color representation of all 
the color indices can be changed, although only a finite number of 
different colors can be obtained (see the manual for the terminal).

\proclaim Input capability: The graphics cursor is a blinking
diamond-crosshair. The user positions the cursor using the arrow keys
and PF1--PF4 keys on his keyboard [Note: NOT the
keyboard of the terminal on which he is plotting, if that is different.]
The arrow keys move the cursor in the appropriate direction; the size of
the step for each keystroke is controlled by the PF1--PF4 keys: PF1
$\Rightarrow$ 1 pixel, PF2 $\Rightarrow$ 4 pixels, PF3 $\Rightarrow$ 16
pixels, PF4 $\Rightarrow$ 64 pixels. [The VT240 terminal has a built-in
capability to position the cursor, but PGPLOT does not use this as it is
not available on the VT125.] The user indicates that the cursor has
been positioned by typing any character other than an arrow or PF1-PF4
key [control characters, \eg, |^C|, and other special characters should
be avoided, as they may be intercepted by the operating system]. 

\proclaim File format:
A REGIS plot file is formatted in records of 80 characters or less, and
has no carriage-control attributes. 
The records are
grouped into ``buffers,'' each of which begins with \esc{\tt Pp} to put the
terminal into graphics mode and ends with \esc{\tt $\backslash$} to put it back into
text mode.  The terminal is in graphics mode only while a buffer is
being transmitted, so a user's program can write to the terminal at any
time (in text mode) without worrying if it might be in graphics mode.
Everything between the escape sequences is REGIS: see the VT125 or VT240
manual for an explanation.  PGPLOT attempts to minimize the number of
characters in the REGIS commands, but REGIS is not a very efficient
format. It does have the great advantage, though, that it can easily be
examined with an editor.   The file may also contain characters outside
the \esc{\tt Pp}$\ldots$ \esc {\tt $\backslash$} delimiters, \eg, escape
sequences to erase the text screen and home the cursor. 

The following escape sequences are used:
\smallskip
\halign{\indent\esc{\tt #}\hfil&\quad -- #\hfil\cr
[2J &Erase entire screen (text)\cr
[H  &Move cursor to home position\cr
Pp  &Enter REGIS graphics mode\cr
$\backslash$ &Leave REGIS graphics mode\cr
}
\smallskip


PGPLOT uses a very limited subset of the REGIS commands supported
by the VT125 and VT240. The following list summarizes the REGIS commands
presently used. 

\medskip\noindent
Initialization: the following standard commands are used to initialize
the device every time a new frame is started; most of these restore a
VT125 or VT240 to its default state, but the screen addressing mode is
nonstandard. 
\smallskip
\halign{\indent\tt #\hfil&\quad-- #\hfil\cr
;		           & resynchronize\cr
W(R)		           & replace mode writing\cr
W(I3)		           & color index 1\cr
W(F3)		           & both bit planes\cr
W(M1)		           & unit multiplier\cr
W(N0)		           & negative off\cr
W(P1)		           & pattern 1\cr
W(P(M2))	           & pattern multiplier 2\cr
W(S0)		           & shading off\cr
S(G1)		           & select graphics plane [Rainbow REGIS]\cr
S(A[0,479][767,0])         & screen addressing, origin at bottom left\cr
S(I0)		           & background dark\cr
S(S1)		           & scale 1\cr
S(M0(L0)(AL0))             & output map section 0 (black)\cr
S(M1(L30)(AH120L50S100))   & output map section 1 (red/dim grey)\cr
S(M2(L59)(AH240L50S100))   & output map section 2 (green/light grey)\cr
S(M3(L100)(AL100))         & output map section 3 (white)\cr
}
\medskip\noindent
Drawing lines: the {\tt P} and {\tt V} commands are used with absolute
coordinates, relative coordinates, and pixel vectors. The {\tt(B)},
{\tt(S)}, {\tt(E)}, and {\tt(W)} modifiers are not used. Coordinates
which do not change are omitted. 
\smallskip
\halign{\indent{\tt #}\hfil&\quad -- #\hfil\cr
P[x,y]		& move to position, \eg \tt\ P[499,0]\cr
V[x,y]		& draw vector to position, \eg \tt\ 
                  V[][767][,479][0][,0]\cr
}
\medskip\noindent
Line attributes: the line style and line color attributes are 
specified with {\tt W} commands, \eg
\smallskip
\halign{\indent{\tt #}\hfil&\quad -- #\hfil\cr
W(P2)		& line style 2\cr
W(I2)		& intensity (color index) 2\cr
}
\smallskip\noindent
and {\tt S} commands are used to change the output
map.  The PGPLOT color indices 0, 1, 2, 3 correspond to output map
sections 0, 3, 1, 2.

\proclaim Obtaining hardcopy: A hardcopy of the plot can be obtained 
using a printer attached to the VT125/VT240 terminal (see the 
instruction manual for the terminal). A plot stored in disk file
can be displayed by copying it to a suitable terminal; \eg, use |TYPE| 
on VMS or |cat| on Unix.


\beginsection VAX Workstations

\proclaim Driver: WSDRIVER, version 4.1 (1989 Jun 7), by S.~C. 
Allendorf.

\proclaim Supported device: This driver should work with all VAX/VMS
workstations running VWS software; it requires the UISSHR shareable
image provided by DEC. 

\proclaim Device type code: |/WS|.

\proclaim Default device name: |PGPLOT|.  Output is always directed to
device |SYS$WORKSTATION|; the ``device name'' provided by the user is
used to label the PGPLOT window. 

\proclaim Default view surface dimensions: Depends on monitor.

\proclaim Resolution: Depends on monitor.

\proclaim Color capability: VAX workstations have 1, 4, or 8 bitplanes.
On 1-plane devices, there are only two colors (background = white,
color index 1 = black). On 4-plane devices, color indices 0--11
are available (4 indices are reserved for text windows and pointers).
On 8-plane systems, color indices 0--249 are available (6 indices
are reserved for text windows and pointers).

\proclaim Input capability: The cursor is controlled by the mouse or the
keypad (arrow keys and PF1--PF4) available on the controlling (DEC-like)
keyboard. The user positions the cursor, and then types any key on the
controlling keyboard.  The mouse buttons are not used. 

\proclaim Notes: The displayed window is deleted when PGEND is executed 
or on program exit. PGPLOT requests confirmation from the user before
deleting the window. Type a carriage-return at the prompt when you are
ready to continue. This makes it impossible to overlay a plot
created by one program on a plot created by another. (The |/APPEND|
qualifier which allows this for other devices has no effect on device
|/WS|.) PGPLOT uses a window which is nominally 11 inches wide by 8.5
inches tall, i.e., the same size as you would get in a hardcopy. If you
prefer a vertical orientation, execute the following command before
running the program: 
\begintt
$ DEFINE PGPLOT_WS_ASPECT PORTRAIT 
\endtt
Substitute |LANDSCAPE| for |PORTRAIT| to revert to horizontal
orientation.

VAXstations can also be used in Tektronix emulation mode. If you run a
process in a Tektronix emulation window, you can use device
specification |/TEK| to tell PGPLOT to plot in Tektronix mode within the
same window. If you run in a VT220 window, you can tell PGPLOT to create
a new Tektronix window and plot in it by giving a device specification
|TK:/TEK|. (|TK:| is the VMS device name of the Tektronix emulator.)
This has one problem: the window will be deleted as soon as your program
calls PGEND or exits; you may need to add a user-prompt in your program
before the call of PGEND. 

\beginsection Sun Workstations

\proclaim Driver: SVDRIV, by Brian M. Sutin (sutin@astro.umd.edu),
1989 May 19. 

\proclaim Supported device: This driver should work with all Sun
workstations running the SunView environment.

\proclaim Device type code: |/SUNVIEW|.

\proclaim Default device name: none. Output is always directed to
the workstation screen.

\proclaim Default view surface dimensions: Depends on monitor.

\proclaim Resolution: PGPLOT uses a square window with $500 \times 500$ 
pixels.

\proclaim Color capability: 32 colors, 16 pre-defined; white background.

\proclaim Input capability: The cursor is controlled by the mouse.
The user positions the cursor, and then types any key on the
controlling keyboard.  The mouse buttons can be used instead of 
keystrokes.

\proclaim Notes: The displayed window is deleted when PGEND is executed 
or on program exit. PGPLOT requests confirmation from the user before
deleting the window. 


\beginsection Grinnell

\proclaim Supported device: Grinnell GMR-270 Image Display System. 

\proclaim Device type code: |/GRinnell|.

\proclaim Default device name: |TV_DEVICE| (a logical name, usually
defined by the system manager).

\proclaim Default view surface dimensions: Depends on monitor.

\proclaim Resolution: The full view surface is $512\times512$ pixels. 

\proclaim Color capability: Color indices 0--255 are supported. The
default color representation is as listed in Chapter 5. The
representation of all color indices can be changed. 

\proclaim Input capability: The graphics cursor is a white cross-hair.
The user positions the cursor using the arrow keys and PF1--PF4 keys on
his terminal keyboard (|SYS$COMMAND|). The arrow keys move the cursor in
the appropriate direction; the size of the step for each keystroke is
controlled by the PF1--PF4 keys: PF1 $\Rightarrow$ 1 pixel, PF2
$\Rightarrow$ 4 pixels, PF3 $\Rightarrow$ 16 pixels, PF4 $\Rightarrow$
64 pixels. The user indicates that the cursor has been positioned by
typing any character other than an arrow or PF1-PF4 key [control
characters, \eg, |^C|, and other special characters should be avoided,
as they may be intercepted by the operating system]. 

\proclaim File format: It is not possible to send Grinnell plots to a 
disk file.

\proclaim Obtaining hardcopy: Not possible.


\beginsection IVAS

\proclaim Supported device: International Imaging Systems IVAS Display 
Processor.

\proclaim Device type code: |/IVAS|.

\proclaim Default device name: |/dev/ga0| (Unix).

\proclaim Default view surface dimensions: Depends on monitor.

\proclaim Resolution: The full view surface is $1024\times1024$ pixels. 

\proclaim Color capability: Color indices 0--15 are supported. The
default color representation is as listed in Chapter 5. The
representation of all color indices can be changed. 

\proclaim Input capability: The graphics cursor is a yellow cross.
The user positions the cursor using the mouse, and 
indicates that the cursor has been positioned by
typing any character other than an arrow or PF1-PF4 key [control
characters, \eg, |^C|, and other special characters should be avoided,
as they may be intercepted by the operating system]. 

\proclaim File format: It is not possible to send IVAS plots to a 
disk file.

\proclaim Obtaining hardcopy: Not possible.


\beginsection Sigma ARGS

\proclaim Supported device: Sigma ARGS color graphic display.

\proclaim Device type code: |/ARgs|.

\proclaim Default device name: |ARGS_DEVICE| (a logical name).

\proclaim Default view surface dimensions: Depends on monitor.

\proclaim Resolution: The full view surface is $512\times512$ pixels. 

\proclaim Color capability: Color indices 0--255 are supported. The
default color representation is as listed in Chapter 5. The
representation of all color indices can be changed. 

\proclaim Input capability: maybe....

\proclaim File format: It is not possible to send ARGS plots to a 
disk file.

\proclaim Obtaining hardcopy: Not possible.


\beginsection Tektronix 4006, 4010

\proclaim Supported device: Tektronix 4006 and 4010 Series Storage Tube 
terminals, and ``emulators.''

\proclaim Device type code: |/TEk4010|.

\proclaim Default device name: |TT| (a logical name, usually equivalent 
to the logged-in terminal).

\proclaim Default view surface dimensions: Depends on monitor.

\proclaim Resolution: The full view surface is nominally 1024 
(horizontal)$\times$ 768 (vertical) pixels, but the actual resolution
varies from device to device.

\proclaim Color capability: None.  Only color index 1 is permitted, and
requests for other color indices are ignored.
It is not possible to change color representation, or to erase by using 
color index 0.

\proclaim Input capability: Maybe....

\proclaim File format: It is not possible to send Tektronix plots to a 
disk file.

\proclaim Obtaining hardcopy: A hardcopy of the plot can be obtained 
using a Tektronix hardcopy unit attached to the terminal.


\beginsection Tektronix 4100 

\proclaim Supported device: Tektronix 4100-series terminals.

\proclaim Device type code: |/TK4100|.

\proclaim Default device name: |TT| (a logical name, usually equivalent 
to the logged-in terminal).

\proclaim Default view surface dimensions: Depends on monitor.

\proclaim Resolution: The view surface is nominally 4096 (horizontal)
$\times$ 3072 (vertical) pixels, but the true resolution depends on
which terminal is used, and is usually much less than this.

\proclaim Color capability: Color indices 0--16 are supported. The
default color representation is as listed in Chapter 5. The
representation of all color indices can be changed. 

\proclaim Input capability: Not yet implemented.

\proclaim File format: It is not possible to send Tektronix plots to a 
disk file.

\proclaim Obtaining hardcopy: It is possible to obtain a hardcopy of the 
plot (in color, even) using a printer attached to the terminal.

 
\beginsection Retrographics 

\proclaim Supported device:  Digital Engineering, Inc., Retrographics
modified VT100 terminal (VT640).

\proclaim Device type code: |/REtro|. 

\proclaim Default device name: |TT:| (the logged-in terminal). 

\proclaim Default view surface dimensions: Depends on monitor. 

\proclaim Resolution: The full view surface is 1024 (horizontal)$\times$ 
780 pixels. 

\proclaim Color capability: Color indices 0 (erase, black) and 1
(bright: usually green) are supported. It is not possible to change
color representation. 

\proclaim Input capability: The graphics cursor is a crosshair across
the entire screen. The user positions the cursor using the four arrow
keys on the keyboard of the Retrographics terminal. ``By striking the
desired directional arrow key, the crosshair will move across the
display screen at the rate of one dot per keystroke. Applying a constant
pressure on an arrow key will cause the crosshair to move at a
continuous rapid rate; releasing the key will stop the crosshair's
movement.'' The user indicates that the cursor has been positioned by
typing any printable ASCII character on the keyboard of the
Retrographics terminal. Most control characters (\eg, |^C|) are
intercepted by the operating system and cannot be used. 

\proclaim File format: It is not possible to send Retro plots to a disk
file. 

\proclaim Obtaining hardcopy: Not possible. 


\beginsection Null Device

\proclaim Supported device: The ``null'' device can be used to suppress
all graphic output from a program.  

\proclaim Device type code: |/Null|.

\proclaim Default device name: None (the device name, if specified, is 
ignored).

\proclaim Default view surface dimensions: Undefined.

\proclaim Resolution: Undefined.

\proclaim Color capability: Color indices 0--255 are accepted.

\proclaim Input capability: None.

\proclaim File format: None.

\proclaim Obtaining hardcopy: Not possible.



\beginsection Canon


\proclaim Supported device:  Canon LBP-8/A2 Laser printer.
Conforms to ISO 646, 2022, 2375 and 6429 specifications.
VDM (graphics) conforms to proposed American National
Standard VDM mode.

\proclaim Device type code:  |/CAnon| (landscape mode), |/VCanon| 
(portrait mode) and |/BCanon| (bitmap in landscape mode only).

\proclaim Default file name:  |PGPLOT.CAN|

\proclaim Default view surface dimensions:  24 cm by 19 cm.

\proclaim Resolution:  300 pixels per inch in both directions.

\proclaim Color capability:  Color indices 0 (erase) and 1 (black) are
supported.  Note, hardware polygon fill is used and colors
0-11 control the fill pattern.

\proclaim Input capability:  None.

\proclaim File format:  Variable length records with Carriage control
of LIST.

\proclaim Obtaining hardcopy:  If printer is connected to a terminal
line (RS-232 option) then printing the file on the corresponding
queue should suffice.  If the printer is connected using
the Centronics interface that appears the to VAX as an
LP device then it is important to ensure that (1) all 8 bit
characters are passed to the printer (2) lines longer than
132 bytes are not truncated, and (3) no extra formatting
commands (e.g. form-feeds) are sent to the printer.
This can be done with the VMS command:
\begintt
$ SET PRINT/PASSALL/LOWER/CR <device>
\endtt
Note, some interface boards have a option to append a carriage
return after a formfeed or LF character, it is necessary
that this be disabled.
The file should be printed with the /PASSALL qualifier i.e.,
\begintt
$ PRINT/PASSALL <filename>
\endtt
Note, SET PRINT/PASSALL and PRINT/PASSALL do not do the
same things and hence PASSALL is required in both locations.

{\it Note:\/} The BCDRIVER  produces
a bitmap that then can be printed on the Canon.  The default
size is 1556 blocks and takes 5 min (parallel) or 15 min (serial
9600 baud) to print.  Thus for simple line graphs CADRIVER
produces much smaller files (typically $<$100 blocks) that
that plot in $<$30 sec.  However, for complex graphs, for
example those obtained with PGGRAY, BCDRIVER will produce
the smaller file and plot faster.  Therefore, it is suggested
that sites with Canon laser printers should support both drivers.



\beginsection 	Colorwriter 6320 Plotter

\proclaim Supported device:	Gould (now Bryans) Colourwriter 6320 or any
		device obeying Gould Plotter Language.

\proclaim Device type code:	|/CW6320|

\proclaim Default device name:	|$PLOTTER1|	(Defined system logical name)

\proclaim Default view surface dimensions:	280mm by 360mm (A3)

\proclaim Resolution:	0.025mm

\proclaim Colour Capability:	Up to 10 pens.Default is pen 1 which is
				picked up on initialization without a call
				to PGSCI.Calls to PGSCI are interpreted as
				the pen number and colours therefore depend
				on how the pens have been loaded into the 
				stalls.If a call is made for a pen higher
				than 10 the selected pen defaults to 1.

\proclaim Input Capability:	Possible but not supported.

\proclaim File format:	Ascii character strings.It is possible to send the
		data to a file which can then be copied to the plotter or
		examined on a terminal.

\proclaim Obtaining hard copy:	PGPLOT has been fixed to send the plot
		directly to the plotter without an intermediate file.



\beginsection Ikon 

\proclaim Supported device:  Digisolve Ikon Pixel Engine

\proclaim Device type code:  |/IKon.|

\proclaim Default device name:  |IKON_DEFAULT| (a logical name).

\proclaim Default view surface dimensions:  Depends on monitor.

\proclaim Resolution:  The full view surface is 1024 by 780 pixels.

\proclaim Color capability: Color indices 0-255 are supported.  The default
representation is listed in Chapter 5 of the PGPLOT manual.  The
representation of all color indices can be changed.

\proclaim Input capability:  

\proclaim File format:  It is not possible to send IKON plots to a disk file.

\proclaim Obtaining hardcopy:  Not possible.



\beginsection Zeta

\proclaim  Supported device:  Zeta 8 Digital Plotter.

\proclaim  Device type code:  |/ZEta|

\proclaim  Default file name:  |PGPLOT.ZET|

\proclaim  Default view surface dimensions:  11 inches by 11 inches.  Current
   version does not allow larger plots although the manual indicates
   plots up to 144 feet are possible.

\proclaim Resolution:  This version is written for the case where the resolution
   switch is set to .025 mm.  Actual resolution depends on thickness
   of pen tip.

\proclaim  Color capability:  Color indices 1 to 8 are supported corresponding
   to pens 1-8.  It is not possible to erase lines.

\proclaim  Input capability:  None.

\proclaim  File format:  Variable length records with Carriage control of LIST.

\proclaim Obtaining hardcopy: On Starlink print the file on the queue associated
   with the Zeta plotter.  If the Plotter is attached to a terminal
   line, then TYPEing the file at the terminal will produce a plot.
   On Starlink:
\begintt
   $ PRINT/NOFEED/QUE=ZETA PGPLOT.ZET
\endtt

   To stop a Zeta plot job, once it has been started, use the buttons
   on the plotter.  Press PAUSE, NEXT PLOT and CLEAR.  Only after
   this sequence is it safe to delete the job from the ZETA Queue.
   Failing to press the NEXT PLOT button will not correctly advance
   the paper.  Failing to press CLEAR but, deleting the current
   job can prevent the following plot from being plotted.

\endchapter
