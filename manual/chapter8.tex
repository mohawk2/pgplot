% PGPLOT manual, 31-May-1988, T. J. Pearson
%------------------------------------------------------------------------
% Copyright (c) 1983, 1984, 1985, 1986, 1987, 1988, 1989 by
% California Institute of Technology.
% All rights reserved.
%------------------------------------------------------------------------

\beginchapter{8}{METAFILES}

\beginsection Introduction

A graphics {\it metafile\/} is a disk file in which a device-independent
representation of a graphics image can be stored. Such a file cannot be
displayed directly on a graphics device, but must first be translated
into the commands appropriate for the specific device using a {\it
metafile translator}.  This may seem like an unnecessary complication,
when PGPLOT can create the device-specific commands directly, but it has
some advantages. Metafiles can be used, for example, to transfer
pictures between two computing sites (they are usually smaller than the
corresponding device-specific files) or for archiving pictures.  It is
sometimes convenient to make a program generate a metafile rather than a
device-specific file so that one can, for example, take a ``quick look''
at the picture on an interactive display before making hard copies. One
may not know what hard-copy devices will be available when the program
runs: for example, if it turns out after your 6-hour batch job has
finished that the Versatec printer has broken, it is nice to be able to
send the plot to some other device without re-running the batch job. 

The metafiles generated by PGPLOT follow the ``GSPC Metafile Proposal''
described in {\it Computer Graphics (A. C. M.)}, volume {\bf 13}, number
3 (August 1979). 

At present, metafiles are available only in the VMS version of PGPLOT, 
not in the UNIX versions.


\beginsection Creating Metafiles

Metafiles may be created using PGPLOT in the same way that any other
plot file is created: the device specification consists of a disk file
name and a device type |/FILE|, for example |PLOT17.GMF/FILE|. The
default file type if none is specified is |.GMF| (for Graphics Meta
File).  Programs which might generate metafile output should not make
any assumptions about the physical scale of the picture; the scale will
vary depending on what device the picture is ultimately plotted on. 


\beginsection Translating Metafiles 

In order to generate graphics output from a metafile, one must use a
{\it Metafile Translator\/} to interpret the device independent metafile
commands. The Metafile Translator (GMFPLOT) provided with PGPLOT uses
the PGPLOT subroutines to generate the device-specific output: thus a
meta\-file may be displayed on any of the devices supported by PGPLOT. (A
metafile may even be translated into another metafile, but this is not
very useful.) To use GMFPLOT, first define a command |PLOT|, say, as
follows; this definition may be included in your |LOGIN.COM| file if you
make extensive use of it: 
\begintt
$ PLOT == "$PGPLOT_DIR:GMFPLOT"
\endtt
The |PLOT| command takes two arguments: the name of the input metafile,
and the device specification for the output. The following sample
commands display a metafile on a Grinnell and on a Versatec printer: 
\begintt
$ PLOT PLOT17.GMF /GR
$ PLOT PLOT17 DEIMOS::LVA0:/VE
\endtt
Again, the default file type for the metafile is |.GMF|.

At present, it is not possible to edit the metafile before display. This
is a facility which might be added one day. Nor is it possible to change
the scale or {\it aspect ratio\/} (ratio of height/width of the display
surface).  When PGPLOT generates a metafile, it does not know the size
or shape of the display surface that it will ultimately be plotted on;
as it has to make some assumption, it assumes that the surface will be
square. The metafile translator displays the metafile in the largest
square available on the output device. Thus a plot which is sent
directly to a terminal may not look exactly the same as one that is
stored in a metafile and subsequently displayed on the terminal. Future
enhancements may allow one to specify the scale and aspect ratio of the
metafile when it is generated. 

\endchapter
