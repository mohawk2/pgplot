% PGPLOT manual, 16-MAY-1989, T. J. Pearson
%------------------------------------------------------------------------
% Copyright (c) 1983, 1984, 1985, 1986, 1987, 1988, 1989 by
% California Institute of Technology.
% All rights reserved.
%------------------------------------------------------------------------

\beginappendix{A}{SUBROUTINE DESCRIPTIONS}

\beginsection Introduction

This appendix includes a classified list of all the PGPLOT subroutines,
and then gives detailed instructions for the use of each routine in
Fortran programs. The subroutine descriptions are in alphabetical order.

\beginsection Arguments

The subroutine descriptions indicate the data type of each argument. When
arguments are described as ``input'', they may be replaced with constants
or expressions in the |CALL| statement, but make sure that the constant
or expression has the correct data type. 

\item{1.} |INTEGER| arguments: these should be declared |INTEGER| or
|INTEGER*4| in the calling program, not |INTEGER*2|. 

\item{2.} |REAL| arguments: these should be declared |REAL| or |REAL*4|
in the calling program, not |REAL*8| or |DOUBLE PRECISION|. 

\item{3.} |CHARACTER| arguments: any valid Fortran |CHARACTER| variable
may be used (declared |CHARACTER*|$n$ for some integer $n$). 


\beginsection Classified List

Note: all routine names begin with the letters ``PG''. Most (but 
unfortunately not all) routine names are six characters or less, to 
conform to Fortran-77 standards.

{\parindent=0pt\obeylines

\beginsub{Control routines}

PGADVANCE -- see PGPAGE
PGASK -- control new page prompting
PGBBUF -- begin batch of output (buffer)
PGBEGIN -- begin PGPLOT, open output device
PGEBUF -- end batch of output (buffer)
PGEND -- terminate PGPLOT
PGPAGE -- advance to a new page or clear screen
PGPAPER -- change the size of the view surface
PGUPDT -- update display

\beginsub{Windows and viewports}

PGBOX -- draw labeled frame around viewport
PGENV -- set window and viewport and draw labeled frame
PGVPORT -- set viewport (normalized device coordinates)
PGVSIZE -- set viewport (inches)
PGVSTAND -- set standard (default) viewport
PGWINDOW -- set window
PGWNAD -- set window and adjust viewport to same aspect ratio

\beginsub{Primitive drawing routines}

PGDRAW -- draw a line from the current pen position to a point
PGLINE -- draw a polyline (curve defined by line-segments)
PGMOVE -- move pen (change current pen position)
PGPOINT -- draw one or more graph markers
PGPOLY -- fill a polygonal area with shading
PGRECT -- draw a rectangle, using fill-area attributes

\beginsub{Text}

PGLABEL -- write labels for x-axis, y-axis, and top of plot
PGMTEXT -- write text at position relative to viewport
PGPTEXT -- write text at arbitrary position and angle
PGTEXT -- write text (horizontal, left-justified)

\beginsub{Attribute setting}

PGSCF -- set character font
PGSCH -- set character height
PGSCI -- set color index
PGSCR -- set color representation
PGSFS -- set fill-area style
PGSHLS -- set color representation using HLS system
PGSLS -- set line style
PGSLW -- set line width

\beginsub{Higher-level drawing routines}

PGBIN -- histogram of binned data
PGCONS -- contour map of a 2D data array (fast algorithm)
PGCONT -- contour map of a 2D data array (contour-following)
PGCONX -- contour map of a 2D data array (non-rectangular)
PGERRX -- horizontal error bar
PGERRY -- vertical error bar
PGFUNT -- function defined by X = F(T), Y = G(T)
PGFUNX -- function defined by Y = F(X)
PGFUNY -- function defined by X = F(Y)
PGGRAY -- gray-scale map of a 2D data array
PGHI2D -- cross-sections through a 2D data array
PGHIST -- histogram of unbinned data

\beginsub{Interactive graphics (cursor)}

PGCURSE -- read cursor position
PGLCUR -- draw a line using the cursor
PGNCURSE -- mark a set of points using the cursor
PGOLIN -- mark a set of points using the cursor

\beginsub{Inquiry routines}

PGQCF -- inquire character font
PGQCH -- inquire character height
PGQCI -- inquire color index
PGQCR -- inquire color representation
PGQFS -- inquire fill-area style
PGQINF -- inquire PGPLOT general information
PGQLS -- inquire line style
PGQLW -- inquire line width
PGQVP -- inquire viewport size and position
PGQWIN -- inquire window boundary coordinates

\beginsub{Utility routines}

PGETXT -- erase text from graphics display
PGIDEN -- write username, date, and time at bottom of plot
PGLDEV -- list available device types
PGNUMB -- convert a number into a plottable character string
PGRND -- find the smallest ``round'' number greater than x
PGRNGE -- choose axis limits
}

\beginsection Subroutine Synopses

The following pages give descriptions of all the PGPLOT subroutines
in alphabetical order.  These descriptions have been extracted from
comments in the Fortran source code.  (For an up-to-date version
of these decriptions, look at the file |PGPLOT.DOC| in the PGPLOT
directory.)

\raggedbottom
\vfill\eject

\input routines
\endchapter
