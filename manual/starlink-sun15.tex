\documentstyle{article} 
\pagestyle{myheadings}

%------------------------------------------------------------------------------
\newcommand{\stardoccategory}  {Starlink User Note}
\newcommand{\stardocinitials}  {SUN}
\newcommand{\stardocnumber}    {15.2}
\newcommand{\stardocauthors}   {D L Terrett}
\newcommand{\stardocdate}      {13 June 1989}
\newcommand{\stardoctitle}     {PGPLOT --- Graphics Subroutine Library}
%------------------------------------------------------------------------------

\newcommand{\stardocname}{\stardocinitials /\stardocnumber}
\markright{\stardocname}
\setlength{\textwidth}{160mm}
\setlength{\textheight}{240mm}
\setlength{\topmargin}{-5mm}
\setlength{\oddsidemargin}{0mm}
\setlength{\evensidemargin}{0mm}
\setlength{\parindent}{0mm}
\setlength{\parskip}{\medskipamount}
\setlength{\unitlength}{1mm}

\catcode`\_=12

\begin{document}
\thispagestyle{empty}
SCIENCE \& ENGINEERING RESEARCH COUNCIL \hfill \stardocname\\
RUTHERFORD APPLETON LABORATORY\\
{\large\bf Starlink Project\\}
{\large\bf \stardoccategory\ \stardocnumber}
\begin{flushright}
\stardocauthors\\
\stardocdate
\end{flushright}
\vspace{-4mm}
\rule{\textwidth}{0.5mm}
\vspace{5mm}
\begin{center}
{\Large\bf \stardoctitle}
\end{center}
\vspace{5mm}

\section{Introduction}

PGPLOT is a high level graphics package for plotting X {\em versus} Y plots,
functions, histograms, bar charts, contour maps and grey-scale images. Complete
diagrams can be produced with a minimal number of subroutine calls, but control
over colour, lines-style, character font, etc.\ is available if required. The
package was written (by Dr T J Pearson of the Caltech astronomy department)
with astronomical applications in mind and has become a {\it de facto} standard
for graphics in astronomy world wide.

The package exists in two version: the original version which uses a low
level graphics package known as GRPCKG which was also written at Caltech,
and a version developed by Starlink, in collaboration with Dr Pearson, which
uses GKS. The two versions have identical subroutine interfaces and
applications can be moved from one version to the other simply by re-linking.
It is the GKS version that is distributed and supported by Starlink.

The main source of information on using PGPLOT is the PGPLOT manual (PGPLOT,
Version 4.7 May 1988, revised June 1989), obtainable
from your site manager or the Starlink software librarian, which describes the
original GRPCKG version. The majority of the manual applies equally to both
versions but when using the GKS version the following sections should be
ignored and the information in this note used instead:
\begin{tabbing}
***\=$\bullet$ Appendix  C**\= \kill
\>$\bullet$ Section 1.3 \>Linking with PGPLOT \\
\\
\>$\bullet$ Section 1.4 \>Graphics Devices \\
\\
\>$\bullet$ Section 2.11 \>Compiling and Running the Program \\
\\
\>$\bullet$ Appendix C \>Installation Instructions \\
\\
\>$\bullet$ Appendix D \>Supported Devices \\
\\
\>$\bullet$ Appendix E \>Writing a Device Handler
\end{tabbing}

\section{Using the GKS version}

There are two ways in which the GKS version of PGPLOT can be used:
     
\begin{enumerate}
\item As a self contained graphics package where {\em all} graphics, including
opening and closing the workstation, is done with PGPLOT. Programs written in
this way can be run with other implementations of PGPLOT. Such programs
are linked with the command:
\begin{quote}\tt
\$ LINK {\em prog},PGPLOT_DIR:GRPSHR/LIB
\end{quote}

\item To plot a picture in the current viewport of an already open GKS
workstation (see section~\ref{viewport}). Programs using PGPLOT in this way must
be linked with the PGPLOT object module library ({\tt PGPLOT_DIR:GRPCKG}), GNS
({\tt GNS_DIR:GNS}) and
GKS as well as any other graphics packages that are being used. 
\end{enumerate}

\section{PGPLOT on Starlink}

On Starlink systems, PGPLOT programs will use the GKS version of the library by
default but the GRPCKG version may also have been installed, either because it
supports some graphics device that the GKS version does not, or because it is
needed by software which contains calls to obsolete GRPCKG routines. The
version being used both when linking and running a program depends on the
definition of the logical name GRPSHR. For the GKS version the definition
will be:
\begin{center}\tt
"GRPSHR" = "PGPLOT_DIR:GRPSHR.EXE"
\end{center}
To use the GRPCKG version the logical name must be re-defined to point the
directory containing the GRPCKG library (ask your system manager if you
don't know where this is). Some software systems (eg.\
FIGARO) may make this re-definition automatically in which case the new
definition must be de-assigned in order to revert to the GKS version.

\section{Graphics Devices}\label{Graphics_Devices}

Any graphics device supported by GKS can be used with PGPLOT and device
names are translated using the graphics name service described in SUN/57
section 2.

If a question mark is typed in response to the prompt from {\tt PGBEGIN}, a
list of those workstation names defined on your system will be listed on the
terminal. 

The device name syntax described in the PGPLOT manual is also
supported; when using this form of device name, the device type is specified
using a GNS workstation name.

On some hard copy devices the output from a PGPLOT program is a file and
some further action (such as printing the file) is required to produce a plot.
If you are unfamiliar with a particular device, consult SUN/83.

\section{Plotting in the current viewport}\label{viewport}

PGPLOT can be used to plot a picture in the current viewport on an already open
GKS workstation. When used in this way, the second argument to {\tt PGBEGIN}
(normally the workstation name) is a GKS workstation identifier (encoded as a
character string). PGPLOT then behaves as if the region of the display surface
defined by the current viewport is a complete workstation. When {\tt PGEND} is
called the workstation is not closed but the state of GKS is restored to what
it was at the time that {\tt PGBEGIN} was called. 

PGPLOT assumes that it has exclusive control over the GKS and so the only
graphics calls allowed between {\tt PGBEGIN} and {\tt PGEND} are PGPLOT
routines and GKS inquiry routines. 

The following simple example is a subroutine that uses PGPLOT to draw an X, Y
plot in an SGS zone.

\begin{verbatim}
      SUBROUTINE XYPLOT (IZONE, X, Y, N, XLO, XHI, YLO, YHI, ISTAT)
*++
*   XYPLOT   Draw X,Y plot in an SGS zone
*
*   Description:
*   
*      Uses PGPLOT to draw an X,Y plot of the real arrays X & Y in the
*      region of the display surface defined be the specified SGS zone
*
*   Input arguments:
*   
*      IZONE    INTEGER         SGS zone identifier
*      X        REAL(N)         X values of data points
*      Y        REAL(N)         Y   "    "    "    "
*      N        INTEGER         Number of data points
*      XLO      REAL            Lower X axis limit
*      XHI      REAL            Higher X  "    "
*      YLO      REAL            Lower Y  "     "
*      YHI      REAL            Higher Y  "    "
*                                 
*   Output arguments:
*   
*      ISTAT    INTEGER         SGS status
*
*   Side effects:
*   
*      The specified SGS zone is selected.
*++
      IMPLICIT NONE
      INTEGER  IZONE, N, ISTAT
      REAL     X(N), Y(N), XLO, XHI, YLO, YHI

      CHARACTER*10 WKID
      INTEGER IWKID

*  Select the specified SGS zone
      CALL SGS_SELZ(IZONE, ISTAT)
      IF (ISTAT.EQ.0) THEN

*     Inquire the GKS workstation identifier of the current zone
         CALL SGS_ICURW(IWKID)

*     Encode workstation id as a character string
         WRITE(UNIT=WKID, FMT='(I10)') IWKID
      
*     Open PGPLOT
         CALL PGBEGIN(0, WKID, 1, 1)

*     Define axis limits
         CALL PGENV(XLO, XHI, YLO, YHI, 0, 0)

*     Plot the data
         CALL PGPOINT(N, X, Y, 2)
      
*     Close down PGPLOT
         CALL PGEND
      END IF
      
      END
\end{verbatim} 

Because other plotting packages may have plotted on the same physical device,
there are some restrictions when using PGPLOT in this way:

\begin{itemize}
\item {\tt PGSIZE} cannot be used.

\item {\tt PGPAGE} will never clear the screen. If the display surface has
been divided into sub-pictures {\tt PGPAGE} will move to the next sub-picture
in the usual way.

\item On devices with fixed colour tables, the default PGPLOT colour table
will not be set up by {\tt PGBEGIN} unless the display surface is empty.

\end{itemize}

\section{Example Programs}

The directory {\tt [STARLINK.LIB.PGPLOT.EXAMPLES]} contains a number of example
programs which demonstrate most of the features of PGPLOT. They can be run
with the command:
\begin{quote}\tt
\$ RUN STARDISK:[STARLINK.LIB.PGPLOT.EXAMPLES]PGEX{\em n}
\end{quote}
where {\em n} is between 1 and 23. All the programs prompt for a device
name. The program source code is stored in the same directory.

\section{Other Differences}
In general the two versions produce identical results when run on the same
device but the following differences should be noted:
\begin{itemize}
\item The GKS version of {\tt PGPAPER} cannot be used to make the display surface
larger than the default size provided by {\tt PGBEGIN}.
\item The GKS version of {\tt PGBEGIN} clears the display surface immediately
instead of waiting until the first vector is plotted.
\end{itemize}

\section{Support}

PGPLOT is Starlink supported software and bugs should be reported through the
usual channels and not by contacting Dr Pearson directly. Problems with the GKS
specific code will be dealt with by Starlink but all changes to the code which
is common to the two versions of PGPLOT must be made in collaboration with Dr
Pearson.

The mixing of calls to PGPLOT and GKS routines is not supported except as
described in section~\ref{viewport}, and neither
version supports the calling of GRPCKG routines directly. Existing programs
that call GRPCKG should be re-written to call the equivalent PGPLOT routines.

\section{Restrictions}

The current version contains the following bugs and restrictions. They will
be removed in a future release.
\begin{itemize}
\item The metafile workstation cannot be used.
\item {\tt PGCONT} and {\tt PGCONX} occasionally leave gaps in contours and 
draw jagged contours.
\item {\tt PGGRAY} occasionally leaves gaps of one pixel between segments of a
grey scale image. Making a tiny adjustment to the viewport will usually
eliminate them.
\end{itemize}

\section{New features}
The following new features are introduced with this release:
\begin{itemize}
\item The ability to use PGPLOT to plot in region of an already open
workstations.
\item A new routine for inquiring the range of colour indices available on a
device ({\tt PGQCOL}).
\item {\tt PGADVANCE} has been replaced by {\tt PGPAGE} so that PGPLOT can be
implemented on system that restrict subroutine names to 8 characters. {\tt
PGADVANCE} has been retained in the library so that existing programs do not
have to be changed.
\item {\tt PGGRAY} will plot a ``pseudo grey scale'' plot on devices without a
true grey scale capability using a random dot algorithm.
\item PGPLOT now uses the GNS system (SUN/57) for translating device
names. This allows device names to be abbreviated.
\end{itemize}
\end{document}
