% PGPLOT manual, 9-May-1988, T. J. Pearson
%------------------------------------------------------------------------
% Copyright (c) 1983, 1984, 1985, 1986, 1987, 1988, 1989 by
% California Institute of Technology.
% All rights reserved.
%------------------------------------------------------------------------

\beginappendix{C1}{INSTALLATION INSTRUCTIONS (VMS)}

\beginsection Introduction

PGPLOT consists of four ``layers'' of subroutines. The subroutines in 
each layer call only subroutines in the same layer or in the next layer 
down:
\smallskip
\item{1.} The top-level PGPLOT subroutines; these routines call only
other PGPLOT routines.
\item{2.} The ``primitive'' PGPLOT routines; these are used by the 
top-level routines and can be called by users' programs. They call
the routines at the next level to perform graphical output.
\item{3.} Device-independent support routines. These are responsible
for things like scaling, windowing, and character-generation.
\item{4.} Device handler routines: these generate graphical output for 
specific device types.  There is one device handler 
subroutine for each supported device type.

Routines at levels 1 and 2 are the only ones that should be referenced 
directly by a user's program. They are all described fully in Appendix~A.

\beginsection Restoring the Save Set

PGPLOT is usually distributed in the form of a VAX/VMS BACKUP save set
on magnetic tape.  This save set contains a directory tree: the main
directory, [PGPLOT], and several subdirectories: [PGPLOT.DRIVERS],
[PGPLOT.EXAMPLES], [PGPLOT.FONTS], [PGPLOT.MANUAL], and [PGPLOT.SOURCE]. 
The save set should be restored to a similar directory tree. The following 
example indicates how the VMS BACKUP command can be used to copy the files into
directories called [USER.PGPLOT] and [USER.PGPLOT.*]. It is a good
idea to make sure that these destination directories either do not exist
or are empty before beginning the BACKUP command. 
\begintt
$ MOUNT/FOREIGN MTA0: PGPLOT
$ BACKUP/LOG/VERIFY MTA0:PGPLOT.BCK/SELECT=[PGPLOT...] -
        DISK:[USER.PGPLOT...]
$ DISMOUNT MTA0:
\endtt

The main directory, [PGPLOT], contains all the files needed to link and 
run PGPLOT programs. The subdirectories contain the source code, example
programs, utility programs, and documentation.

The directory [PGPLOT] includes:
\smallskip
\item{1.}Object code: the object-module library GRPCKG.OLB, the
shareable image GRPSHR.EXE, and the shareable-image symbol-table
library GRPSHR.OLB. 
\item{2.}Command procedures: LOGICAL.COM and MAKEHELP.COM.  
These procedures are described below.
\item{3.}Documentation: PGPLOT.DOC is the text of the subroutine synopses in 
Appendix~A.  PGPLOT.HLB is a VMS-format HELP library containing the
same information. 
\item{4.}Miscellaneous: DOCUMENT.FOR, DOCUMENT.EXE (used in creating
documentation); GRFONT.DAT (binary representation of the PGPLOT
character set); GMFPLOT.FOR and GMFPLOT.EXE (meta\-file translator);
PGATTRIB.FOR (a Fortran include file defining symbolic names for the
various PGPLOT attribute codes).

The directory [PGPLOT.DRIVERS] contains source code for device-handler 
subroutines that are {\it not\/} included in the executable version of 
PGPLOT as distributed. Instructions for incorporating these device 
handlers in the executable version are given below. 
Adding support for a new device type requires the addition of a new 
device handler subroutine. No other changes to PGPLOT are required. For 
details, consult Appendix~E.

The directory [PGPLOT.EXAMPLES] contains a number of programs for 
demonstrating and testing PGPLOT. The source code is in files PGEX1\-.FOR, 
PGEX2\-.FOR, \etc, and executable code in PGEX1.EXE, \etc. The command 
procedure EXCOMP.COM can be used to recompile and link one or all of the 
example programs.

The directory [PGPLOT.FONTS] contains some utility programs for 
manipulating the binary file containing character definitions (font 
information). This directory can be deleted if you do not want to
edit the character set.

The directory [PGPLOT.MANUAL] contains the text of this PGPLOT manual
(in \TeX\ format) and Fortran programs for creating the figures. This 
directory can be deleted if you do not wish to print copies of the 
manual, or if you do not have the \TeX\ document formatting program.

The directory [PGPLOT.SOURCE] contains the complete source code for
PGPLOT, including the device handlers built in to the distribution
version. PGPLOT.FOR contains the Fortran source code for routines in
levels 1 and 2; all these routines have names beginning with PG.
GRPCKG.FOR, GRCHAR.FOR, GRSYMB.FOR, and GRVMS.FOR contain the source code for
routines in level~3. The file GRVMS.FOR contains {\it 
system-dependent\/} routines which will have to be rewritten for a 
different operating system. The various files *DRIVER.FOR contain the source
code for the level-4 device handlers. All the routines in level~3 have
names beginning with GR. The files PGPLOT.INC and GRPCKG1.INC are
referenced by VAX Fortran INCLUDE statements in PGPLOT.FOR and
GRPCKG.FOR, respectively. The command procedures ADD.COM, BUILD\-.COM,
COMPILE.COM, and NEWGE.COM are used to rebuild PGPLOT from the source
code (see below). 


\beginsection Logical Names

A number of VMS logical names must be defined before PGPLOT can be used.
The procedure LOGICAL.COM will define these logical names, but it may be
necessary to modify it slightly. A possible definition is the following:
\begintt
$ DEFINE/SYSTEM PGPLOT_DIR   SYS:[PGPLOT]
$ DEFINE/SYSTEM LNK$LIBRARY  PGPLOT_DIR:GRPSHR.OLB
$ DEFINE/SYSTEM PGPLOT_TYPE  PRINTRONIX
$ DEFINE/SYSTEM PGPLOT_FONT  PGPLOT_DIR:GRFONT.DAT
$ DEFINE/SYSTEM GRPSHR       PGPLOT_DIR:GRPSHR.EXE
\endtt
It is not essential that they be ``system'' logical names.
\smallskip
\item{1.} |PGPLOT_DIR|.  This is a logical name pointing to the directory
containing the PGPLOT files.  The name need not be |PGPLOT_DIR|; 
another name can be chosen, so long as the other logical names are 
modified accordingly.
\item{2.} |LNK$LIBRARY|.  This is optional.  It tells the VMS linker
to automatically scan GRPSHR.OLB for subroutine references. If 
|LNK$LIBRARY| is already defined, choose another name like 
|LNK$LIBRARY_1| (see the VAX/VMS Linker manual).
\item{3.} |PGPLOT_TYPE|.  This is optional.  It provides a default 
device type for use by PGPLOT when no device type is provided explicitly 
by the user.
\item{4.} |PGPLOT_FONT|.  This is essential.  Whenever a PGPLOT program is 
run, it attempts to read the character-set definition from the file 
defined by this logical name. (If it fails, the program will continue, 
but requests to plot text or graph markers will be ignored.)
\item{5.} |GRPSHR|.  This logical name must be defined at run-time
if the shareable-image version of PGPLOT is used (see below).


\beginsection The Shareable Image

Programs using the PGPLOT library can be linked in two different ways: 
\begintt
$ LINK EXAMPLE,PGPLOT_DIR:GRPSHR/LIB
$ LINK EXAMPLE,PGPLOT_DIR:GRPCKG/LIB
\endtt
Usually the first library (|GRPSHR.OLB|) should be used. When this
library is used, the subroutines are not included in the |.EXE| file,
but are fetched from a {\it shareable image\/} when you execute the 
|RUN|
command. This makes the |.EXE| file much smaller, and means that the
program need not be relinked when changes are made to the graphics
subroutines; but the |.EXE| file can only be run on a machine which has
a copy of the shareable image. If the second library (|GRPCKG.OLB|) is
used, the subroutines are included in the |.EXE| file and the program
can be run on any VAX computer running a compatible version of VMS.  In
general, the shareable image is to be preferred. 


\beginsection Recompiling PGPLOT

To recompile PGPLOT, proceed as follows.
\smallskip
\item{1.} In directory |[PGPLOT.SOURCE]|: make sure that the source code 
for all the device handlers you wish to include is in this directory
(files |*DRIVER.FOR|). Device handlers that you do not wish to include 
should be moved to directory |[PGPLOT.DRIVERS]|.
\item{2.}Execute the DCL
command procedure |COMPILE.COM|. This compiles all the Fortran files in 
the directory, and creates a new version of the
object-module library |GRPCKG.OLB|.  See the comments in the procedure for
more information. 
\item{3.}The shareable image, |GRPSHR.EXE|, and shareable-image symbol-table
library, |GRPSHR.OLB|, can be rebuilt from |GRPCKG.OLB| by executing the
command procedure |BUILD.COM|.  Again, see the comments in the procedure
for more information. (Note: do not modify the symbol table defined in 
|BUILD.COM|; if you do, the resulting programs will not be portable.)
\item{4.}You can now delete all the object module files (|*.OBJ|) if you 
wish.
\item{5.}The new version of PGPLOT consists of three files: |GRPCKG.OLB|,
|GRPSHR.EXE|, and |GRPSHR.OLB|. It is possible to test the new version by 
redefining logical name |GRPSHR| to point to the new |GRPSHR.EXE|. If you 
are satisfied with the new version, rename (or copy) these three files 
into the main directory, |[PGPLOT]|, superseding the old version.


\beginsection Recompiling the Example Programs

The procedure |EXCOMP.COM| in directory |[PGPLOT.EXAMPLES]| can be used to 
compile and link any of the example programs:
\begintt
$ @EXCOMP 7
$ @EXCOMP
\endtt
The first command compiles PGEX7.  The second compiles {\it all\/}
the example programs.


\beginsection Rebuilding the Documentation Files

The files |PGPLOT.DOC| and |PGPLOT.HLB| contain documentation extracted 
from the PGPLOT source code. The files can be rebuilt by executing the
procedure |MAKEHELP.COM| in directory |[PGPLOT]|. This procedure also
updates the \TeX\ version of the documentation included in Appendix~A of 
the manual.


\beginsection Printing the Manual

The \TeX\ source-code for the PGPLOT manual can be found in the
subdirectory |[PGPLOT.MANUAL]| in a number of files with type |.TEX|.
These files can be used to make a printable copy of the manual if you
have the \TeX\ text-formatting program: 
\begintt
$ TEX PGPLOT
\endtt 
The file |PGPLOT.TEX| contains \TeX\ |\input| commands that reference the
other |.TEX| files.  Please do not modify the \TeX\ files: send your
comments and bug reports to the author. 

The figures in the manual were generated using PGPLOT subroutines.  The 
Fortran programs used to generate them can also be found in the 
|[PGPLOT.MANUAL]| directory.


\beginsection Adding a new PGPLOT routine

The author is happy to receive contributions of new high-level routines 
for inclusion in PGPLOT.  Some general guidelines should be observed when
preparing such routines:
\item{1.} The routine should be written (as far as possible) in standard
Fortran-77.
\item{2.} The routine should call only other PGPLOT routines, or special
support routines provided along with it. It should not use routines from
other libraries (especially commercial ones!).  
Note that if it is to be installed in the shareable library, it cannot use 
common  blocks for communication with the program that calls it.  It 
may, if necessary, use common blocks to communicate with other PGPLOT
routines, but this is discouraged.
\item{3.} It should be modular: that is, it should avoid, as far as 
possible, making assumptions about the environment it is to run in, or 
modifying that environment.  In particular: if it changes any attribute
(\eg, color index) it should save the previous value of that attribute
(obtained by calling a |PGQ...| routine) and reset it before exiting;
it should not change the window or viewport, but if it must, it 
should restore the old values before exiting; it should not call 
|PGBEGIN| or |PGEND|.
\item{4.} It should call routine |PGBBUF| at the beginning, and |PGEBUF|
before exiting, in order to buffer its output.
\item{5.} It should provide a generally useful function rather than a 
highly specific one, with as many parameters as possible accessible to 
the user (\eg, as arguments) rather than fixed in the program. 

\endchapter
