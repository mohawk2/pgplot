% PGPLOT manual,  29-JUN-1989, T. J. Pearson
%------------------------------------------------------------------------
% Copyright (c) 1983, 1984, 1985, 1986, 1987, 1988, 1989 by
% California Institute of Technology.
% All rights reserved.
%------------------------------------------------------------------------

\beginappendix{F}{CALLING PGPLOT FROM A C PROGRAM}

\beginsection Introduction

It is possible to call PGPLOT routines from a program written in C.
The methods for calling a Fortran subroutine from a C program are
operating-system dependent, and indeed it is impossible in some
systems. The details are rather complicated, and if there is a 
demand, we could create a C-callable library that hides these details 
from the user. Examples of the same program written in Fortran-77, 
VAX/VMS C, and Convex UNIX C are included below.

\beginsection VMS

The following is a prescription for calling PGPLOT subroutines from
a C program on a VAX running VMS.  
\smallskip
\item{1.}All arguments (except arrays and character strings) are passed by
	address using the |&| operator. As you cannot take the address of a 
	constant, constants must be passed in dummy variables.
\item{2.} |INTEGER| arguments correspond to C type |long int| or |int|.
\item{3.} |REAL| arguments correspond to C type |float|.
\item{4.} |CHARACTER| arguments correspond to C character strings, but they must
	be passed by descriptor. The VAX-C manual explains how to do this using
	the |$DESCRIPTOR| macro for fixed strings; it is a little more complicated
	for variable strings. 
\item{5.} Note that the backslash character (|\|) must be escaped (|\\|).

\beginsection Convex UNIX

The following is a prescription for calling PGPLOT subroutines from
a C program on the Convex. It may also work on other Berkeley UNIX 
systems.
\smallskip
\item{1.} The main program must be called |MAIN__()| instead of |main()|. 
	I don't know why this is so, but it must have to do with the
	initialization of the Fortran library.
\item{2.} Use the C compiler to compile the program, but use |fc| to load it. 
	This ensures that the Fortran system libraries are scanned.
\item{3.} All PGPLOT subroutine names must be typed in lower case, with an 
underscore appended, e.g., |pgbegin_()|, |pgend_()|.
\item{4.} All arguments (except arrays and character strings) are passed by
address using the |&| operator. As you cannot take the address of a 
constant, constants must be passed in dummy variables.
\item{5.} |INTEGER| arguments correspond to C type |long int|.
\item{6.} |REAL| arguments correspond to C type |float|.
\item{7.} For |CHARACTER| arguments, pass a pointer to a C character string,
and add the length of the string (number of characters) as an extra
|long int| argument at the end of the argument list.
\item{8.} Note that the backslash character (|\|) must be escaped (|\\|).
\smallskip
Example:
\begintt
cc -c example1.c
fc -o example1 example1.o -lpgplot
example1
\endtt

\pageinsert
\eightpoint
\begintt
C ---------------------------------------------------------------------
C Demonstration program for PGPLOT (Fortran version).
C ---------------------------------------------------------------------
C
      PROGRAM PGDEMO
C
      INTEGER I
      REAL XS(5),YS(5), XR(100), YR(100)
      DATA XS/1.,2.,3.,4.,5./
      DATA YS/1.,4.,9.,16.,25./
C
C Call PGBEGIN to initiate PGPLOT and open the output device; PGBEGIN
C will prompt the user to supply the device name and type.
C
      CALL PGBEGIN(0,'?',1,1)
C
C Call PGENV to specify the range of the axes and to draw a box, and
C PGLABEL to label it. The x-axis runs from 0 to 10, and y from 0 to 20.
C
      CALL PGENV(0.,10.,0.,20.,0,1)
      CALL PGLABEL('(x)', '(y)', 'PGPLOT Example 1 - y = x\u2')
C
C Mark five points (coordinates in arrays XS and YS), using symbol
C number 9.
C
      CALL PGPOINT(5,XS,YS,9)
C
C Compute the function at 60 points, and use PGLINE to draw it.
C
      DO 10 I=1,60
          XR(I) = 0.1*I
          YR(I) = XR(I)**2
   10 CONTINUE
      CALL PGLINE(60,XR,YR)
C
C Finally, call PGEND to terminate things properly.
C
      CALL PGEND
C
      END
\endtt
\vfill
\endinsert

\pageinsert
\eightpoint
\begintt
/* ---------------------------------------------------------------------
 * Demonstration program for PGPLOT called from C [VMS].
 *----------------------------------------------------------------------
 */

#include descrip

main()
{
    int i;
    static float xs[] = {1.0, 2.0, 3.0, 4.0, 5.0 };
    static float ys[] = {1.0, 4.0, 9.0, 16.0, 25.0 };
    float xr[100], yr[100];
    long dummy, nx, ny;
    float xmin, xmax, ymin, ymax;
    long just, axis, n, symbol;
    $DESCRIPTOR(device, "?");
    $DESCRIPTOR(xlabel, "(x)");
    $DESCRIPTOR(ylabel, "(y)");
    $DESCRIPTOR(toplabel, "PGPLOT Example 1 - y = x\\u2");
/*
 * Call PGBEGIN to initiate PGPLOT and open the output device; PGBEGIN
 * will prompt the user to supply the device name and type.
 */
    dummy = 0;
    nx = 1;
    ny = 1;
    pgbegin(&dummy, &device, &nx, &ny);
/*
 * Call PGENV to specify the range of the axes and to draw a box, and
 * PGLABEL to label it. The x-axis runs from 0 to 10, and y from 0 to 20.
 */
    xmin = 0.0;
    xmax = 10.0;
    ymin = 0.0;
    ymax = 20.0;
    just = 0;
    axis = 1;
    pgenv(&xmin, &xmax, &ymin, &ymax, &just, &axis);
    pglabel(&xlabel, &ylabel, &toplabel);
/*
 * Mark five points (coordinates in arrays XS and YS), using symbol
 * number 9.
 */
    n = 5;
    symbol = 9;
    pgpoint(&n, xs, ys, &symbol);
/*
 * Compute the function at 60 points, and use PGLINE to draw it.
 */
    n = 60;
    for (i=0; i<n; i++)
        {
        xr[i] = 0.1*i;
        yr[i] = xr[i]*xr[i];
        }
    pgline(&n, xr, yr);
/*
 * Finally, call PGEND to terminate things properly.
 */
    pgend();
}
\endtt
\vfill
\endinsert

\pageinsert
\eightpoint
\begintt
/* ---------------------------------------------------------------------
 * Demonstration program for PGPLOT called from C [Convex UNIX].
 *----------------------------------------------------------------------
 */

MAIN__()
{
    int i;
    static float xs[] = {1.0, 2.0, 3.0, 4.0, 5.0 };
    static float ys[] = {1.0, 4.0, 9.0, 16.0, 25.0 };
    float xr[100], yr[100];
    long dummy, nx, ny, just, axis, n, symbol;
    float xmin, xmax, ymin, ymax;
/*
 * Call PGBEGIN to initiate PGPLOT and open the output device; PGBEGIN
 * will prompt the user to supply the device name and type.
 */
    dummy = 0;
    nx = 1;
    ny = 1;
    pgbegin_(&dummy, "?", &nx, &ny, 1L);
/*
 * Call PGENV to specify the range of the axes and to draw a box, and
 * PGLABEL to label it. The x-axis runs from 0 to 10, and y from 0 to 20.
 */
    xmin = 0.0;
    xmax = 10.0;
    ymin = 0.0;
    ymax = 20.0;
    just = 0;
    axis = 1;
    pgenv_(&xmin, &xmax, &ymin, &ymax, &just, &axis);
    pglabel_("(x)", "(y)", "PGPLOT Example 1 - y = x\\u2", 3L, 3L, 27L);
/*
 * Mark five points (coordinates in arrays XS and YS), using symbol
 * number 9.
 */
    n = 5;
    symbol = 9;
    pgpoint_(&n, xs, ys, &symbol);
/*
 * Compute the function at 60 points, and use PGLINE to draw it.
 */
    n = 60;
    for (i=0; i<n; i++)
        {
        xr[i] = 0.1*i;
        yr[i] = xr[i]*xr[i];
        }
    pgline_(&n, xr, yr);
/*
 * Finally, call PGEND to terminate things properly.
 */
    pgend_();
}
\endtt
\vfill
\endinsert

\endchapter
