% PGPLOT manual, 9-Jun-1988, T. J. Pearson
%------------------------------------------------------------------------
% Copyright (c) 1983, 1984, 1985, 1986, 1987, 1988, 1989 by
% California Institute of Technology.
% All rights reserved.
%------------------------------------------------------------------------

\beginchapter{4}{PRIMITIVES}

\beginsection Introduction

Having selected a view surface and defined the viewport and the window,
we are ready to draw the substance of the image that is to appear within
the viewport.  This chapter describes the most basic routines, called
{\it primitives}, that can be used for drawing elements of the image.  
There are four different sorts of primitive:  {\it lines, graph-markers,
text}, and {\it area fill}.  Chapter 5 explains how to change the 
{\it attributes\/} of these primitives, \eg, color, line-style, text font;
and Chapter 6 describes some higher-level routines that simplify
the composition of images that would require a large number of calls to 
the primitive routines.

The primitive routines can be used in any combination and order after
the viewport and window have been defined.  They all indicate where the
primitive is to appear on the view surface by specifying world
coordinates. See the subroutine descriptions in Appendix A
for more details. 

\beginsection Clipping

The primitives are ``clipped'' at the edge of the viewport: any parts 
of the image that would appear outside the viewport are suppressed.  The 
various primitives behave slightly differently. A {\it line\/} is
clipped where it crosses the edge of the viewport. A {\it graph marker\/}
is plotted if the center (the point marked) lies within or on the edge 
of the viewport; otherwise it is suppressed. {\it Text\/}, which is 
usually used for annotation, is not clipped (except at the edge of the 
view surface. A {\it filled area\/} is clipped at the edge of the
viewport. 


\beginsection Lines

The primitive line-drawing routine is |PGLINE|. This draws one 
or more connected straight-line segments (generally called a {\it 
polyline\/} in computer graphics).  It has three arguments: the number 
(|N|) of points defining the polyline, and two arrays (|XPTS| and 
|YPTS|) containing the world $x$ and $y$-coordinates of the points. 
The polyline consists of $N-1$ straight-line segments connecting 
points 1--2, 2--3, $\ldots$, $(N-1)$--$N$:
\begintt
CALL PGLINE (N, XPTS, YPTS)
\endtt

The two routines |PGMOVE| and |PGDRAW| are even more primitive than 
|PGLINE|, in the sense
that any line graph can be produced by calling these two routines alone.
In general, |PGLINE| should be preferred, as it is more
modular. |PGMOVE|
and |PGDRAW| are provided for those who are used to Calcomp-style plotting
packages.  |PGMOVE| moves the plotter ``pen'' to a specified point, without 
drawing a line (``pen up''). It has two arguments: the world-coordinates of the
required new pen position.  |PGDRAW| moves the plotter ``pen'' from its 
current position (defined by the last call of |PGMOVE| or |PGDRAW|) to a new 
point, drawing a straight line as it goes (``pen down'').  The above call to
|PGLINE| could be replaced by the following:
\begintt
CALL PGMOVE (XPTS(1), YPTS(1))
DO I=2,N
    CALL PGDRAW (XPTS(I), YPTS(I))
END DO
\endtt


\beginsection Graph Markers

A Graph Marker is a symbol, such as a cross, dot, or circle, drawn on a graph 
to mark a specific point. Usually the symbol used will be chosen to be 
symmetrical with a well-defined center. The routine |PGPOINT| draws one 
or more graph markers (sometimes called a {\it polymarker\/}).
It has four arguments: the number 
(|N|) of points to mark, two arrays (|XPTS| and 
|YPTS|) containing the world $x$ and $y$-coordinates of the points, and 
a number (|NSYM|) identifying the symbol to use:
\begintt
CALL PGPOINT (N, XPTS, YPTS, NSYM)
\endtt
The symbol number can be: $-1$, to draw a dot of the smallest possible 
size (one pixel); 0--31, to draw any one of the symbols in Figure
\the\chapnum.1; or 33--127, to draw the corresponding ASCII character
(the character is taken from the currently selected text font); or 
$>127$, to draw one of the Hershey symbols from Appendix~B. The 
Fortran |ICHAR| function can be used to obtain the ASCII value; \eg, to 
use letter $F$:
\begintt
CALL PGPOINT (1, 0.5, 0.75, ICHAR('F') )
\endtt

\pageinsert
\insertplot{fig41.ps}{7.5}{4.5}{2.0}{1.75}{1}{0}
\smallskip
\centerline{{\bf Figure \the\chapnum.1}\quad Standard Graph Markers.}
\endinsert


\beginsection Text

The Text primitive routine is used for writing labels and titles
on the image.  It converts an internal computer representation of the
text (ASCII codes) into readable text.  The simplest routine for
writing text is |PGTEXT|, which writes a horizontal character string 
starting at a specific $(x,y)$ world coordinate position, \eg,
\begintt
CALL PGTEXT (X, Y, 'A text string')
\endtt
|PGTEXT| is actually a simplified interface to the more general
primitive routine |PGPTEXT|, which allows one to
change orientation and justification of the text, \eg,
\begintt
CALL PGPTEXT (X, Y, 45.0, 0.5, 'A text string')
\endtt
writes the text at an angle of $45\deg$ to the horizontal, centered
at $(x,y)$ (see Appendix A).

Both |PGTEXT| and |PGMTEXT| require the position of the text string
to be specified in world coordinates.  When annotating a graph, it is 
usually more convenient to position the text relative to the edge of the
viewport, rather than in world-coordinate space.  The routine |PGMTEXT|
(see Appendix A) is provided for this, and |PGLABEL| provides a simple 
interface to |PGMTEXT| for the normal job of annotating an $(x,y)$ 
graph.

The appearance of text can be altered by specifying a number of {\it
attributes}, described in the next chapter.  In particular, the
character size and character font can be changed.  Figure \the\chapnum.2 
illustrates some of the possibilities.

\pageinsert
\insertplot{fig42.ps}{7.5}{4.5}{2.0}{1.75}{1}{0}
\smallskip
\centerline{{\bf Figure \the\chapnum.2}\quad Text examples.}
\endinsert

To include one of the graph marker symbols in a text string, use the
Fortran |CHAR| function, \eg,
\begintt
CALL PGTEXT (X, Y, 'Points marked with '//CHAR(17))
\endtt

The routine |PGPTEXT| (and all the PGPLOT routines which call
it, \eg, |PGTEXT|, |PGLABEL|) allows one to include {\it escape 
sequences\/}
in the text string to be plotted. These are character-sequences that are
not plotted, but are interpreted as instructions to change font, draw
superscripts or subscripts, draw non-ASCII characters (\eg, Greek
letters), \etc\  All escape sequences start with a backslash character
(|\|). The following escape sequences are defined (the letter following
the |\| may be either upper or lower case): 

\noindent |\u| -- start a superscript, or end a subscript;

\noindent |\d| -- start a subscript, or end a superscript (note that
|\u| and |\d| must always be used in pairs); 

\noindent |\b| -- backspace (\ie, do not advance text pointer after
plotting the previous character);

\noindent |\\| -- backslash character (|\|);

\noindent |\A| -- \AA ngstr\"om symbol (\AA);

\noindent |\g|$x$ -- greek letter corresponding to roman letter $x$;

\noindent |\fn| -- switch to Normal font (1);

\noindent |\fr| -- switch to Roman font (2);

\noindent |\fi| -- switch to Italic font (3);

\noindent |\fs| -- switch to Script font (4);

\noindent |\(|$n$|)| -- character number $n$ (1 to 4 decimal digits);
the closing parenthesis may be omitted if the next character is neither
a digit nor ``)''.  This makes a number of special characters (\eg,
mathematical, musical, astronomical, and cartographical symbols)
available.  See Appendix~B for a list of available characters. 


Greek letters are obtained by |\g| followed by one of the following
upper-case and lower-case letters:
$$\vbox{\halign{\hfil#\quad&&\ \hfil$\mit#$\hfil\cr
use:&A&B&G&D&E&Z&Y&H&I&K&L&M&N&C&O&P&R&S&T&U&F&X&Q&W\cr
for:&A&B&\Gamma&\Delta&E&Z&H&\Theta&I&K&\Lambda&M&N&\Xi&O&\Pi&P&\Sigma&T&\Upsilon&\Phi&X&\Psi&\Omega\cr
 or:&a&b&g&d&e&z&y&h&i&k&l&m&n&c&o&p&r&s&t&u&f&x&q&w\cr
for:&\alpha&\beta&\gamma&\delta&\epsilon&\zeta&\eta&\theta&\iota&\kappa&
\lambda&\mu&\nu&\xi&o&\pi&\rho&\sigma&\tau&\upsilon&\phi&\chi&
\psi&\omega\cr
}}$$
Use uppercase letters for uppercase Greek,
lowercase for lowercase. Example: |\gh| is $\theta$, lowercase ``theta''.


\beginsection Area Fill

The Area Fill primitive allows the programmer to shade the interior of
an arbitrary polygonal region.  The appearance of the primitive is
controlled by attributes {\it fill area style\/} and {\it color index\/}
(see Chapter 5).  An area is specified by the set of vertices of the
polygon. 

The routine |PGPOLY| is used to fill an area. It has three arguments:
the number (|N|) of vertices defining the polygon, and two arrays
(|XPTS| and |YPTS|) containing the world $x$ and $y$-coordinates
of the vertices: 
\begintt
CALL PGPOLY (N, XPTS, YPTS)
\endtt
If the polygon is not convex, it may not be obvious which points in the
image are inside the polygon. PGPLOT assumes that a point is inside the
polygon if a straight line drawn from the point to infinity intersects
an odd number of the polygon's edges.

For the special case of a {\it rectangle\/} with edges parallel to the 
coordinates axes, it is better to use routine |PGRECT| instead of 
|PGPOLY|; this routine will use the hardware rectangle-fill capability 
if available. |PGRECT| has four arguments: the $(x,y)$ world coordinates 
of two opposite corners (note the order of the arguments):
\begintt
CALL PGRECT (X1, X2, Y1, Y2)
\endtt

\endchapter
