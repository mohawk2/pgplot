% PGPLOT manual, 31-May-1988, T. J. Pearson
%------------------------------------------------------------------------
% Copyright (c) 1983, 1984, 1985, 1986, 1987, 1988, 1989 by
% California Institute of Technology.
% All rights reserved.
%------------------------------------------------------------------------

\beginchapter{5}{ATTRIBUTES}

\beginsection Introduction

The appearance of the primitive elements of a graphical image
(lines, graph-markers, text, and area-fill) can be changed by specifying
{\it primitive attributes}.  The attributes, and the corresponding
routines for changing them, are: 

{\it Color Index\/} and {\it Color Representation:}\quad |PGSCI|, 
|PGSCR|, and |PGSHLS|.

{\it Line Style:}\quad |PGSLS|.

{\it Line Width:}\quad |PGSLW|.

{\it Character Height:}\quad |PGSCH|.

{\it Character Font:}\quad |PGSCF|.

{\it Fill-area Style:}\quad |PGSFS|. 

The routines to change attributes can be freely intermixed with the
PGPLOT drawing routines. Once an attribute has been changed by a call to
the appropriate routine, it remains in effect for all subsequent
plotting until it is changed again.  In addition to the routines that
set attributes (|PGSxx|) there are routines for determining the current
value of each attribute (|PGQxx|). These make it possible to write
subroutines which change attribute values temporarily but restore the
old attributes before returning to the calling program. 


\beginsection Color Index

This attribute affects all the primitives: lines, graph-markers, text,
and area-fill, and is controlled by two subroutines: |PGSCI| and
|PGSCR|. 

Devices differ considerably in their ability to draw in more than one 
color.  On most hardcopy devices, the default color is black on a white 
background, while on most CRT devices, it is white (or green) on a black
background.  Color is selected using an integer parameter called the
{\it color index}.  Color index 1 is the default color, and color
index 0 is the background color.  The number of different color indices
available depends on the device. On most monochrome devices, only color 
indices 0 and 1 are available, while some color CRT devices may permit color
indices from 0 to 255.  On some monochrome devices, color index can be 
used to select different brightnesses (intensities).

Color index 0, the background color, can be used to ``erase'' elements
from a picture by overwriting them with color index 0.  Note that not 
all devices are capable of this: e.g., Tektronix storage-tube 
terminals and pen-plotters cannot erase part of a displayed picture.

To select a new color index for subsequent plotting, use routine |PGSCI|
(Set Color Index).

Appendix~D lists the capabilities of the devices for plotting in color
and variable intensity.  The default color index is 1; all devices
accept this.  Most devices also accept color index 0 (erase), and
several accept color index up to 15 or more. The maximum color index is
the number of different colors that can be displayed at once. Some
devices permit the assignment of colors to color indices to be changed
(by calling |PGSCR|, see below). 


\beginsection Color Representation

Each color index has an associated {\it Color Representation}, which
defines the associated color and intensity. Color Representation may
be expressed by a set of three numbers, either the Hue, Lightness, and
Saturation $(H,L,S)$ components or the Red, Green, and Blue $(R,G,B)$ 
components. $(R,G,B)$ are quantities in the range 0.0--1.0, with 1.0
being maximum intensity; if $R=G=B$ the color is a shade of gray. In the
$(H,L,S)$ system, hue is a cyclic quantity expressed as an angle in
the range 0--360, while $L$ and $S$ are in the range 0.0--1.0. 

Table \the\chapnum.1 shows how the color indices are defined
when PGPLOT is started (not all are available on all devices).
The default assignments of colors to color indices can be changed with
routine |PGSCR|, which permits one to specify the $(R,G,B)$ values for
any color index, or |PGSHLS|, which permits one to specify the $(H,L,S)$
values. Note that color-index 0, the background color, can be 
redefined in this way.

The effect of |PGSCR| is device-dependent. One some devices, it will be
ignored. On others, (\eg, Grinnell, VT125) it will change the color of
lines which have already been drawn in the specified color index, while
on others (\eg, pen plotters) it will only affect lines drawn after the
call of |PGSCR|. 


\topinsert
\centerline{\bf Table \the\chapnum.1\quad
Default Color Representations}
\medskip
\centerline{\vbox{\tabskip=0pt \offinterlineskip
\def\tablerule{\noalign{\hrule}}
\halign {& \vrule#\tabskip=1em plus 2em&
  \hfil\strut#\hfil& \vrule#&
       #\hfil& \vrule#& 
  \hfil#&      \vrule#&
  \hfil#& \vrule#\tabskip=0pt\cr
\tablerule
height2pt&\omit&&\omit&&\omit&&\omit&\cr
&{\it Color}&&   \omit&&\omit&&\omit&\cr
&{\it Index}&& {\it Color}&& $(H,L,S)$\hfil&& $(R,G,B)$\hfil&\cr
height2pt&\omit&&\omit&&\omit&&\omit&\cr
\tablerule
height2pt&\omit&&\omit&&\omit&&\omit&\cr
& 0&&	Black (background)&&	  0, 0.00, 0.00&&	0.00, 0.00, 0.00&\cr
& 1&&	White (default)&&	  0, 1.00, 0.00&&	1.00, 1.00, 1.00&\cr
& 2&&	Red&&			120, 0.50, 1.00&&	1.00, 0.00, 0.00&\cr
& 3&&	Green&&			240, 0.50, 1.00&&	0.00, 1.00, 0.00&\cr
& 4&&	Blue&&			  0, 0.50, 1.00&&	0.00, 0.00, 1.00&\cr
& 5&&	Cyan (Green + Blue)&&	300, 0.50, 1.00&&	0.00, 1.00, 1.00&\cr
& 6&&	Magenta (Red + Blue)&&	 60, 0.50, 1.00&&	1.00, 0.00, 1.00&\cr
& 7&&	Yellow  (Red + Green)&&	180, 0.50, 1.00&&	1.00, 1.00, 0.00&\cr
& 8&&	Red + Yellow (Orange)&&	150, 0.50, 1.00&&	1.00, 0.50, 0.00&\cr
& 9&&	Green + Yellow&&	210, 0.50, 1.00&&	0.50, 1.00, 0.00&\cr
&10&&	Green + Cyan&&		270, 0.50, 1.00&&	0.00, 1.00, 0.50&\cr
&11&&	Blue + Cyan&&		330, 0.50, 1.00&&	0.00, 0.50, 1.00&\cr
&12&&	Blue + Magenta&&	 30, 0.50, 1.00&&	0.50, 0.00, 1.00&\cr
&13&&	Red + Magenta&&		 90, 0.50, 1.00&&	1.00, 0.00, 0.50&\cr
&14&&	Dark Gray&&		  0, 0.33, 0.00&&	0.33, 0.33, 0.33&\cr
&15&&	Light Gray&&		  0, 0.66, 0.00&&	0.66, 0.66, 0.66&\cr
&16--255&&	Undefined&&                    &&	                &\cr
height2pt&\omit&&\omit&&\omit&&\omit&\cr
\tablerule
}}}
\endinsert


\beginsection Line Style

Line Style can be, \eg, solid, dashed, or dotted. The
attribute affects only lines, not the other primitives. It is controlled
by subroutine |PGSLS|. The default line style is a full, unbroken
line. To change the line style, use routine |PGSLS|. Line style is
described by an integer code: 
\item{1} -- full line,
\item{2} -- long dashes,
\item{3} -- dash-dot-dash-dot,
\item{4} -- dotted,
\item{5} -- dash-dot-dot-dot.


\beginsection Line Width

Line Width affects lines, graph-markers, and text. A thick-nibbed pen is
simulated by tracing each line with multiple strokes offset in the
direction perpendicular to the line.  The line width is specified by the
number of strokes. The default width is one stroke, and the maximum that
may be specified is 201. The exact appearance of thick lines is
device-dependent---it depends on the resolution of the device---but on
hardcopy devices (\eg, QMS Lasergrafix, Versatec) PGPLOT attempts to
make the line-width increment equal to 0.005~inches. Requesting a
line-width of 10, say, should give lines that are approximately
1/20~inch thick. To change the line width, use routine |PGSLW|. 


\beginsection Character Height

Character Height affects graph-markers and text.  Character height is 
specified as a multiple of the default character height; the default
character height one-fortieth of the height or width of the view surface
(whichever is less).  To change the character height, use routine 
|PGSCH|.


\beginsection Character Font

Character Font affects text only. Four fonts are available. The default
font (1) is simple and is the fastest to draw; the others should only be
used for presentation plots on a high-resolution device (\eg, Versatec
or laser printer).  To change the character font, use routine |PGSCF|; 
it is also possible to change the font temporarily by using 
escape sequences (see \S 4.4). The
font is defined by an integer code: 
\item{1} -- normal (simple) font (default),
\item{2} -- roman font,
\item{3} -- italic font,
\item{4} -- script font.


\beginsection Fill-Area Style

Fill-Area Style can be hollow (only the outline of the polygon is
drawn), or solid. The attribute may be extended in future to allow
hatching and other patterns.  To change the fill-are style, use routine
|PGSFS|.  The style is defined by an integer code:
\item{1} -- solid fill (default),
\item{2} -- hollow (outline only).

\endchapter
