% PGPLOT manual,  14-JUL-1989, T. J. Pearson
%------------------------------------------------------------------------
% Copyright (c) 1983, 1984, 1985, 1986, 1987, 1988, 1989 by
% California Institute of Technology.
% All rights reserved.
%------------------------------------------------------------------------

\beginchapter{1}{INTRODUCTION}

\beginsection PGPLOT

PGPLOT is a Fortran subroutine package for drawing simple scientific
graphs on various graphics display devices. It was originally developed
for use with astronomical data reduction programs in the Caltech Astronomy
department.

This manual is intended for the Fortran programmer who wishes to write a
program generating graphical output.  For most applications, the program
can be device-independent, and the output can be directed to the
appropriate device at run time.  The output device is described by a
``device specification,'' discussed below. The programmer can build a
specific device specification into the program, but it is better to make
this a parameter which the user of the program can supply. 

All the examples in this manual use standard Fortran-77. PGPLOT itself 
is written mostly in standard Fortran-77, with a few non-standard, 
system-dependent subroutines. At Caltech, it runs under the VAX/VMS,
Convex-UNIX, and Sun-UNIX operating systems.


\beginsection This Manual

This manual is intended both as a tutorial introduction to PGPLOT and as 
a reference manual.  The remainder of this chapter describes some 
fundamentals: how to include the PGPLOT library in your program, and the
types of graphic devices that PGPLOT can use.

Chapter 2 is tutorial: it presents a Fortran program for drawing a
graph using the minimum number of PGPLOT subroutines, and explains what
each of these subroutines does.  After reading this chapter, you should 
be able to write your own PGPLOT program, although it may be helpful to 
refer to the individual subroutine descriptions in Appendix~A.

The basic features of PGPLOT are introduced in Chapters 3, 4, and 5.
Chapter 3 explains the positioning and scaling of plots on the page,
Chapter 4 describes the basic (``primitive'') routines for drawing 
lines, writing text, drawing graph markers, and shading areas, and
Chapter 5 describes the routines for changing the ``attributes'' of 
these primitives: color, line-style, line-width, text font, etc.

Chapter 6 describes some ``high level'' routines that use the primitive 
routines to build up more complicated pictures: \eg, function plots,
histograms, bar charts, and contour maps.

Chapter 7 describes PGPLOT's capabilities for ``interactive'' graphics,
whereby the user of the PGPLOT program can control its action with a
cursor, joystick, mouse, etc.

Chapter 8 describes the use of ``metafiles''. A metafile is a disk file
in which a device-independent representation of a graphics image can be
stored. A translation program allows the image to be displayed on
any supported device. 

There are six appendices.  Appendix~A is a list of all the PGPLOT
routines, with detailed instructions for their use.  Appendix~B shows
the complete set of PGPLOT characters and symbols that can be used for
annotating graphs.  Appendix~C is intended for those who want to install
PGPLOT on another machine. Appendix~D gives details of the devices
supported by PGPLOT.  Appendix~E provides instructions for programmers 
who want to extend PGPLOT to support other devices. Appendix~F describes 
how PGPLOT subroutines can be called from a program written in the C 
language.


\beginsection Using PGPLOT

In order to use PGPLOT subroutines, you will need to link your
program with the graphics subroutine library. 

\beginsub{VAX/VMS} On the Caltech Astronomy VAX computers, the graphics
subroutine library is scanned automatically by the |LINK| command, so
the following sequence of instructions suffices to compile, link, and
run a graphics program |EXAMPLE.FOR|: 
\begintt
$ FORTRAN EXAMPLE
$ LINK EXAMPLE
$ RUN EXAMPLE
\endtt
On other VMS computers, the automatic search of the graphics library may not
occur.  You will then need to include the graphics library explicitly by
using a |LINK| commands like the following:
\begintt
$ LINK EXAMPLE,PGPLOT_DIR:GRPSHR/LIB
\endtt
The PGPLOT subroutines are not included in your |.EXE| file, but are
fetched from a {\it shareable image} when you execute the |RUN| command.
This makes the |.EXE| file much smaller, and means that the program need
not be relinked when changes are made to the graphics subroutines; but
the |.EXE| file can only be run on a machine that has a copy of the
shareable image and is running a compatible version of VAX/VMS. 
For more information, see Appendix~C.

\beginsub{Unix} The following assumes that the PGPLOT 
library |libpgplot.a| has been installed in a standard location where 
the loader can find it. To compile, link, and run a graphics program
|example.f|: 
\begintt
fc -o example example.f -lpgplot
example
\endtt
Unlike the VMS version, the PGPLOT routines are included in the 
executable file.


\beginsection Graphics Devices

Graphics devices fall into two classes: devices which produce a hardcopy
output, usually on paper; and interactive devices, which usually display
the plot on a TV monitor. Some of the interactive devices allow
modification to the displayed picture, and some have a movable cursor
which can be used as a graphical input device. There is also a ``null
device,'' to which unwanted graphical output can be directed. 
Hardcopy devices are not used interactively.  One must first create a
disk file and then send it to the appropriate device with a print or
copy command. Consult Appendix~D (or your System Manager) to determine
the appropriate device-specific command. 

A PGPLOT graphical output device is described by a ``device
specification'' that consists of two parts, separated by a slash (/):
the {\it device name\/} or {\it file name}, and the {\it device type}.

\beginsub{Device name} The 
device name or file name is the name by which the output device is
known to the operating system. For most hardcopy devices, this should be
the name of a disk file, while for interactive devices, it should be the
name of a device of the appropriate type; in both cases, the name should
be specified according to the syntax of the operating system in use. If
the device or file name includes a slash (|/|), enclose the name in
double quotation marks (|"|). If the device name is omitted from the
device specification, a default device is used, the default depending on
the device type (see Appendix~D). In Unix, device and file names are 
case-sensitive.

\beginsub{Device type} The
device type tells PGPLOT what sort of graphical device it is.
Appendix~D lists the device types available at the time of writing,
together with the names by which they are known to PGPLOT. If the device
type is omitted, a system-dependent default type is assumed (this is the
value of the ``environment variable'' |PGPLOT_TYPE|, and on Phobos and
Deimos it is ``Printronix''). The device type is not case-sensitive: you 
can use uppercase or lowercase letters, or a mixture of the two.

\beginsub{Examples (VMS)}

\noindent
Tektronix 4006/4010 terminal: |TTA4/TEK| (device |_TTA4:|).

\noindent
Grinnell image display: |/GRIN|.

\noindent
Disk file, Printronix format: |SYS$SCRATCH:PLOT.DAT/PRIN|.

\noindent
Disk file, Versatec format, with the output file on a different DECnet 
node: |DEIMOS::XPLOT.DAT/PRIN|.

\noindent
Disk file in default format in default directory: |PGPLOT.LIS|.

\beginsub{Examples (Unix)}

\noindent
Tektronix 4006/4010 terminal: |/TEK| (the logged-in terminal).

\noindent
IVAS image display: |/IVAS|.

\noindent
Disk file, Printronix format: |"/scr/tjp/plot.dat"/PRIN|.

\noindent
Disk file in default format in default directory: |pgplot.lis|.

\beginsection Environment variables

The run-time behavior of PGPLOT can be modified by defining one or more
{\it environment variables}.  The variables have names which begin with
|PGPLOT_|. In VMS, they are logical names; in Unix, they are Unix
environment variables. 

To set the value of a variable in VMS (DCL):
\begintt
$ DEFINE PGPLOT_ENVOPT VG
\endtt
In Unix (csh):
\begintt
setenv PGPLOT_ENVOPT VG
\endtt
Quotation marks may be required around the value (double-quotes in VMS,
single quotes in Unix) to prevent interpretation of special characters
by the command interpreter. 

To unset a variable in VMS or Unix:
\begintt
$ DEASSIGN PGPLOT_ENVOPT
unsetenv PGPLOT_ENVOPT
\endtt

The following are some of the environment variables currently in use:

\item{$\bullet$} |PGPLOT_ENVOPT|: this variable provides additional
options for the |PGENV| subroutine (see description in Appendix~A). 

\item{$\bullet$} |PGPLOT_FONT|: the name of the binary file containing
character font digitization, \eg, |PGPLOT_FONT| =
|"/usr/tjp/grfont.dat"|. 

\item{$\bullet$} |PGPLOT_IDENT|: if this variable is defined (with any
value), the user name and time are written at the lower right corner of
the plot by routine |PGEND| (hardcopy devices only), \eg, |PGPLOT_IDENT|
= |YES|. 

\item{$\bullet$} |PGPLOT_TYPE|: the plot type to be used in PGPLOT when
a device specification omits the type, \eg, |PGPLOT_TYPE| = |QMS|. 

\item{$\bullet$} |PGPLOT_BUFFER|: controls buffering (see Chapter~7). 

\endchapter
